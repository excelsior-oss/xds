\chapter{Using the dialog mode}
\label{dialog}

\section{Introduction}
\label{dialog:intro}

XD is a text mode application and usually operates in a Win32 console window. 
It may also be started in fullscreen mode. In the following text, {\em screen} refers
to either fullscreen or windowed console containing XD.

The different types of information displayed by the debugger
are collected in separate {\em windows}. There are five
kinds of windows in XD:
\begin{itemize}
\item menus
\item the {\bf Program} window
\item information windows (or just windows)
\item dialogs
\item message boxes
\end{itemize}
The top line of XD screen is occupied by the {\bf Main} menu, which
provides access to other menus. The rest of the screen is shared by
the {\bf Program} window and {\em information windows}.
All those windows are overlapped and may be freely moved, resized, and placed
above or beneath each other.

The {\bf Program} window contains the source text and/or the disassembled
code of the program being debugged (the {\it debuggee}).
It is frameless and may not be closed or hidden.

Information windows are used to display various data about the current state
of the debuggee, such as watch expressions, CPU registers, call stack, etc.
They have frames and may be displayed, hidden, or closed at any time.

A {\em dialog} is a window which {\em pops up} when XD prompts you for an information
it needs to complete a requested action, such as a name of a variable to examine,
or an expression to watch. Dialog windows are {\em modal}: another window
may not be activated while the dialog is displayed and you have to explicitly
confirm or cancel the action you requested.

The debugger displays error, warning, and informational messages in
{\it message boxes}. They are similar to dialogs, but the only
action you may perform when a message box is displayed is to close it.

\subsection{Using menus and shortcuts}
\label{dialog:intro:menus}

To select an item from the {\bf Main} menu, do one of the following:
\begin{itemize}
\item Click an item with the left mouse button.
\item Hold down {\bf Alt} and press a highlighted character of an item.
\item Press {\bf F10}, choose an item with left/right arrow keys,
and press {\bf Enter} or the down arrow key.
\end{itemize}
This will cause one of the {\em pulldown menus} to be displayed. To select
an item from a pulldown menu, do one of the following:
\begin{itemize}
\item Click an item with the left mouse button.
\item Press a highlighted character (holding down {\bf Alt} is optional).
\item Choose an item with up/down arrow keys and press {\bf Enter}.
\end{itemize}
Finally, there are {\em pop-up menus} available in most information windows.
To activate a pop-up menu, click an information window with the {\em right} mouse button
or press {\bf Ctrl+F10}. Then select an item as in a pulldown menu.

The most often used menu commands have {\em shortcuts} - key combinations
which may be used to activate a menu command without selecting it from a menu.
Shortcuts are shown to the right of menu items.
% You may change the default
% shortcuts. See \ref{dialog:customization:shortcuts} for more information.

\subsection{Using information windows}
\label{dialog:intro:info}

Most information windows contains lists of items, such as global variables
or watch expressions. When a window becomes current, a cursor is displayed
over one of its items. The following keys may be used to move the cursor:

\KeyListBegin{8 cm}
\KeyDef{Left/Right}     {scroll window contents horizontally}
\KeyDef{Up/Down}        {move the cursor one line up/down}
\KeyDef{PgUp/PgDn}      {move the cursor one page up/down}
\KeyDef{Ctrl+PgUp/PgDn} {move the cursor to the first/last item of the list}
\KeyDef{Ctrl+Home/End}  {move the cursor to the top/bottom of the window}
\KeyListEnd

An item may also be selected by clicking it with the left or right mouse button.
In the latter case, a \Ref{pop-up menu}{dialog:intro:menus} will also be displayed.

Information windows which contain lists of program entities,
for instance, a list of all modules or a list of local variables
provide {\em locators} for quick cursor positioning.

To move the cursor to a certain item, start typing item's name on the keyboard.
The characters you type will appear on the bottom line of a window frame
(the locator), and the cursor will position on the first item which name 
begins with those characters. You may use the {\bf Tab} key
to quickly expand the current value of a locator.

Once the cursor appears near the desired item, you may use arrow keys
again to complete positioning. Character case is ignored during search, and
case of characters in a locator always match the currently selected item.

\subsection{Using dialogs}
\label{dialog:intro:dialogs}

A dialog is a modal window which pops up when you initiate a certain
action which requires parameters, such as variable name or address.

XD dialogs behave very similar to Windows dialogs.

\subsection{Managing windows}
\label{dialog:intro:windows}

At any moment during a debugging session, there is exaclty one
window which accepts user input - the {\it current window}.
It is the topmost and is displayed with a double frame instead of
a single\footnote{The frameless {\bf Program} window is marked as current
by changing its header colour.}.

To select a window to become current, click it or press {\bf Ctrl+Tab}
to cycle through windows. You may also select a window from the
{\bf Windows} menu, which always contains a list of all open windows.
You may need to scroll it if there are too many windows open.

To move or resize the current window using the keyboard,
press {\bf Ctrl+W}. A bold frame will be drawn around the window.
Use arrow keys to move the window.
Holding down {\bf Ctrl} while pressing arrow keys will force
the window to change its size instead of position.
To leave the move/resize mode, press {\bf Ctrl+W} or {\bf Enter}.

With the mouse, drag the top border of a window to move it or
drag any other border or a corner to resize.

To close the current window, press {\bf Ctrl+F4} or click the
\verb'[x]' sign in its top right corner. It is also possible to close
a window by selecting it in the {\bf Windows} menu using arrow keys
and pressing {\bf Del}. The {\bf Program} window may not be closed.

\subsection{Refreshing the screen}
\label{dialog:intro:refresh}

If there is any garbage on XD screen, press {\bf Ctrl+E}
or select {\bf Refresh screen} from the {\bf File} menu to have
all windows repainted.

\subsection{Viewing batch mode output}
\label{dialog:intro:log}

If the debugger was invoked in the \See{batch mode}{Chapter }{batch}
and then switched to the dialog mode with the help of the
\See{{\bf DIALOG} command}{}{batch:DIALOG}, you may wish to look at the
last lines of the log. To do this, press {\bf "/"} or select
{\bf Show output} from the {\bf File} menu.

\subsection{Smart session select}
\label{dialog:intro:sss}

Your program is likely to require user interaction. It would be convenient
to have the session in which it is executed automatically brought
to foreground (selected) during execution, and XD session to be selected
whenever the program is stopped. However, in case of
\See{step execution modes}{}{dialog:executing} usage that would cause
undesirable flickering.

XD uses a bit smarter approach: after resuming the debuggee execution,
it waits for a small amount of time before selecting the debuggee session.
If control returns to the debugger prior to timeout, its session remains
selected. The \See{{\bf Delay to select debuggee} option}{}{dialog:options}
allows you to adjust the timeout period.

When your program is stopped, you may temporarily bring it to foreground
by pressing {\bf "\bs{}"} or by selecting {\bf Show debuggee} from the
{\bf File} menu. After a timeout, the XD session will be automatically
selected again. A timeout does not occur if you select the debuggee
using an operating system mechanism.

% ??? different timeout for this case?

\section{Files used by XD}
\label{dialog:sysfiles}

The following files are used by the debugger:

\begin{tabular}{ll}
\tt xd.red & redirection file \\
\tt xd.cfg & default session settings \\
\tt xd.hlp & on-line help \\
\tt xd.msg & debugger messages \\
% \tt *.kbd  & keyboard mappings \\
\tt *.ses  & session settings
\end{tabular}

The debugger uses a redirection file to search for all other files
the same way the compiler does, except that source files are searched via
redirection only if the \See{{\bf Strip path from source name} option}{}{dialog:options}
is switched ON.

The \verb'xd.cfg', \verb'xd.hlp', and \verb'xd.msg' files, if not found via redirection,
are expected to be found in the directory in which XD executable file, \verb'xd.exe',
resides. The \verb'xd.msg' file is essential; the debugger will display an error message
and terminate if it fails to locate it on startup.

The configuration file contains default values of options, window sizes, positions,
and colors.

% \verb'*.kbd' files contain alternate keyboard mappings. See \ref  % !!! Remove this alternative from the menu.
% for more information.

Upon termination of a dialog debugging session,
the debugger stores its settings in a session file, provided that
the \See{{\bf Save debug session layout} option}{}{dialog:options}
is set ON. A session file name matches the debugged executable name,
and has the extension \verb'.ses'. Next time you debug that
executable, XD will restore option values and window layout from
the session file.

% !!! New in 1.1:

\See{Watch expressions}{}{dialog:data:watches}, if any, are also saved 
in session files.

\section{Loading a program}
\label{dialog:load}

To have a program which you want to debug automatically loaded when
you invoke the debugger, pass its executable name and optional arguments
on XD command line (see \ref{start:invoking}).
If you do not provide this information,
XD will display the {\bf Load Program} dialog upon startup.

\ifcomment
\begin{figure}[h]
\begin{center}
\ \drawing{dlgload}\
\end{center}
\caption{The {\bf Load Program} dialog}
\end{figure}
\fi

To load a program via the {\bf Load Program} dialog:
\begin{enumerate}
\item Type the name of the executable in the {\bf Program name}
      entry field or select the {\bf Browse} button to select it using
      a file dialog.
\item If you want any command line arguments to be passed to the
      debuggee, type them in the {\bf Arguments} entry field.
\item Select the {\bf OK} pushbutton or press {\bf Enter}.
\end{enumerate}

If you decided not to load a program, select the {\bf Cancel}
pushbutton or press {\bf Esc}.

% !!! write about PATH and xd.red.

To load another program during a debugging session, select
{\bf Load Program} from the {\bf File} menu or press {\bf F3}.

\section{Navigating through your program}
\label{dialog:navigating}

XD displays the program you are debugging in the {\bf Program}
window. This window is automatically opened upon program load
and remains on the screen until you finish the debugging session.

The {\bf Program} window is frameless. The top line is a header in which
information about the current position in the code is displayed.
The two leftmost columns are reserved for \See{breakpoint marks}{}{dialog:breaks}
and are not scrolled horizontally. The remaining area of the window is used
to display the program source and/or the disassembled code.

\subsection{Cursors}
\label{dialog:navigating:cursors}

The {\bf Program} window contains two cursors, displayed in different colors.

When the debuggee is stopped, the {\it execution cursor} is positioned
over the line or instruction about to be executed. It may not be moved manually.

The {\it user cursor} is intended for navigation. You may freely move it
through the source text or code, which is scrolled, if neccessary,
so that the user cursor always stays visible. It is also used to specify a
breakpoint position. See \ref{dialog:breaks} for more information.

When the two cursors coincide, only the execution cursor is displayed.
The user cursor is moved to the position of the execution cursor
when the debuggee execution is interrupted, for instance, when
a step command is executed or a \Ref{breakpoint}{dialog:breaks} is hit.
You may also quickly move the user cursor to the position of the execution
cursor by pressing {\bf Ctrl+Home} or selecting {\bf Home} from the
{\bf Search} menu.

\subsection{View modes}
\label{dialog:navigating:modes}

The {\bf Program} window supports three view modes: {\bf Source}, {\bf Disassembly},
and {\bf Mixed}.

\ifcomment
\begin{figure}[htb]
\begin{center} \ \drawing{source}\ \end{center}
\caption{Source mode}
\end{figure}
\fi

In the {\bf Source} mode, the {\bf Program} window contains source code of
one of the modules which constitute your program. Its name is
displayed in the header along with the number of the source line
currently selected by the user cursor.

Two different colors are used to display executable
and non-executable lines, unless the
\See{{\bf Code highlight} option}{}{dialog:options}
is set to {\bf No}. When the user cursor is positioned
over an executable line, a corresponding address and a
procedure name are displayed in the window header.

\ifcomment
\begin{figure}[htb]
\begin{center} \ \drawing{disasm}\ \end{center}
\caption{Disassembly mode}
\end{figure}
\fi

In the {\bf Disassembly} mode, the code of your program is displayed
as assembly instructions. A name of a procedure to which the
instruction under the user cursor belongs and offset of this
instruction from the procedure's entry point are shown in the window
header. In this mode, you may scroll through the entire code segment,
while in two other modes module boundaries may not be crossed.

\ifcomment
\begin{figure}[htb]
\begin{center} \ \drawing{mixed}\  \end{center}
\caption{Mixed mode}
\end{figure}
\fi

The {\bf Mixed} mode is similar to the {\bf Disassembly} mode, with
addition that the correspondent source line is displayed
above each chunk of disassembled code. The source lines, however,
are treated as comments; the user cursor jumps over them as you
move it, so it is not possible to step execute an entire line or set
a breakpoint on it.

{\bf Note:} In the {\bf Source} and {\bf Mixed} modes,
only one of the program's modules is displayed in the {\bf Program} window.
Use the \See{{\bf Modules} window}{}{dialog:navigating:modules}
to display another module.

After loading a program, XD always enters the {\bf Source} mode,
unless there is no debug info at the execution point.
If a source file can not be found, the window is filled
with lines containing the corresponding message.

To toggle view modes, select a corresponding item from the {\bf Code} menu,
from the pop-up menu, or use the following shortcuts:

\KeyListBegin{11 cm}
\KeyDef{Alt+1} {switch to the {\bf Source} mode}
\KeyDef{Alt+2} {switch to the {\bf Disassembly} mode}
\KeyDef{Alt+3} {switch to the {\bf Mixed} (source+disassembly) mode}
\KeyListEnd

\subsection{Moving around in the Program window}
\label{dialog:navigating:moving}

The following keys move the user cursor through the contents
of the {\bf Program} window:

\KeyListBegin{8 cm}
\KeyDef{Left/Right}     {scroll window contents horizontally}
\KeyDef{Up/Down}        {move the user cursor one line up/down}
\KeyDef{PgUp/PgDn}      {move the user cursor one page up/down}
\KeyDef{Ctrl+PgUp/PgDn} {move the user cursor to the top/bottom of the current module}
\KeyDef{Ctrl+Home/End}  {move the user cursor to the top/bottom of the window}
\KeyDef{Home/End}       {show the current line beginning/end}
\KeyDef{Ctrl+Up/Down}   {move the user cursor to the previous/next procedure in the current module}
\KeyDef{Ctrl+H}         {move the user cursor to the execution cursor position}
\KeyDef{Ctrl+G}         {move the user cursor to a source line by its number}
\KeyDef{Ctrl+A}         {move the user cursor to a certain address}
\KeyDef{Ctrl+F}         {find a string}
\KeyDef{Ctrl+N}         {repeat last find}
\KeyListEnd

Some of the key combinations listed above are shortcuts for the
\See{{\bf Search} menu}{}{dialog:navigating:search} commands.

In addition, the following information windows may help you to quickly move to
certain points in the code:

\begin{tabular}{lll}
\bf Window & \bf May be used to \\
\hline
\bf \See{Modules}   {}{dialog:navigating:modules}    & display another module \\
\bf \See{Procedures}{}{dialog:navigating:procedures} & display a procedure in the current module \\
\bf \See{Publics}   {}{dialog:navigating:publics}    & display a public procedure \\
\bf \See{Call Stack}{}{dialog:executing:callstack}   & display a call chain element \\
\bf \See{Breaks}    {}{dialog:breaks:window}         & display code where a breakpoint is set \\
\bf \See{Components}{}{dialog:navigating:components} & switch to another component \\
\bf \See{Threads}   {}{dialog:executing:threads}     & switch to another thread
\end{tabular}

\subsection{The Search menu}
\label{dialog:navigating:search}

The {\bf Search} menu commands may be used to quickly move to certain
positions in windows. Most of them are available in the {\bf Program}
window only.

\begin{description}
\item[{\bf Find}] \mbox{} \\
  Searches for a string in the currently displayed source module.
  Shortcut: {\bf Ctrl+F}
\item[{\bf Find next}] \mbox{} \\
  Finds the nexts occurence of a string.
  Shortcut: {\bf Ctrl+N}
\item[{\bf Find prev}] \mbox{} \\
  Finds the previous occurence of a string.
\item[{\bf Next procedure}] \mbox{} \\
  Moves the user cursor to the beginning of the next procedure.
  Shortcut: {\bf Ctrl+Down}
\item[{\bf Prev. procedure}] \mbox{} \\
  Moves the user cursor to the beginning of the previous procedure.
  Shortcut: {\bf Ctrl+Up}
\item[{\bf Goto line}] \mbox{} \\
  Moves the user cursor to a given line in the displayed module.
  Shortcut: {\bf Ctrl+G}
\item[{\bf Home}] \mbox{} \\
  Moves the user cursor to the current execution cursor position.
  Shortcut: {\bf Alt+Home}
\item[{\bf Goto address}] \mbox{} \\
  Moves the user cursor to a certain address.
  Shortcut: {\bf Ctrl+A}
\end{description}

\subsection{The Components window}
\label{dialog:navigating:components}

In the {\bf Components} window, the debugger displays the list
of executable components (EXE and DLLs) of your program.
It can be opened by selecting {\bf Components} from the
{\bf Code} menu.

When your program is first loaded into the debugger, its EXE file
becomes the current component. To choose another component,
double-click it in the {\bf Components} window, or select it and
press {\bf Enter}.

% !!! New in version 1.1:

The {\bf Components} window contains two marks similar to
\See{cursors}{}{dialog:navigating:cursors}: one for the currently
displayed component and another for the component in which
debuggee execution stopped.

Use the pop-up menu to toggle display of full paths to components
on or off.

\subsection{The Modules window}
\label{dialog:navigating:modules}

The {\bf Modules} window contains the list of modules which constitute
the current \Ref{component}{dialog:navigating:components}.
To open it, select {\bf Modules} from the {\bf Code}
menu or press {\bf Ctrl+M}. To display another module in the
{\bf Program} window, double-click its name, select it using arrow
keys and press {\bf Enter}, or select {\bf Show code} from the pop-up menu.

By default, only modules for which debug information is available
are displayed. Switch the \See{{\bf Show all modules} option}{}{dialog:options}
ON or select {\bf Show all} from the pop-up menu to display
all modules.

% !!! New in version 1.1:
The module that is currently displayed in the 
\Ref{{\bf Program} window}{dialog:navigating} and the module that contains
the instruction pointer have marks in the first column. The color
of each mark matches the color of the respective
\Ref{cursor}{dialog:navigating:cursors}.

To display global variables of a particular module in a separate window
(see \ref{dialog:data:module}), press {\bf Ctrl+V} or select 
{\bf Show variables} from the pop-up menu.

\subsection{The Procedures window}
\label{dialog:navigating:procedures}

All procedures of the current module are listed in the {\bf Procedures}
window, which may be opened by selecting {\bf Procedures} from
the {\bf Code} menu or by pressing {\bf Ctrl+L}. To position the user cursor
on a procedure, double-click its name or select it using arrow
keys and press {\bf Enter}.

\subsection{The Publics window}
\label{dialog:navigating:publics}

The {\bf Publics} window contains all public names belonging to your
program's code. Double-clicking or pressing {\bf Enter} on a name
causes the correspondent source or disassembled code to be
displayed in the {\bf Program} window.


\section{Executing your program}
\label{dialog:executing}

The debugger provides different methods to execute your program. You
may use the {\it continuous execution commands} to quickly move the execution
point to a certain position in the code, to watch execution flow,
or to wait for an event, such as variable access. The {\it step commands}
allow you to monitor your program execution.

If the {\bf Program} window is in the {\bf Source} mode, step commands
are performed in terms of source lines, otherwise --- in terms of CPU
instructions.

{\bf Note:} Execution of your program may be interrupted earlier than
implied by the command used to initiate it:
\begin{itemize}
\item if your program terminates
\item if an exception is raised in your program
\item if a breakpoint is hit
\item if a break is activated
\end{itemize}
You may also explicitly interrupt your program and have control
returned to XD.

The execution commands are collected in the {\bf Run} menu:

\begin{description}
\item[{\bf Run}] \mbox{} \\
    Start or resume program execution at full speed.
\item[{\bf Execution trace}] \mbox{} \\
    Start or resume line-by-line program execution.
\item[{\bf Step into}] \mbox{} \\
    Execute a single source or assembly line, stepping into a procedure call,
    if there is one.
\item[{\bf Step over}] \mbox{} \\
    Execute a single source or assembly line without stepping into procedure calls.
\item[{\bf Step out}] \mbox{} \\
    Execute until return from the current procedure.
\item[{\bf Run to cursor}] \mbox{} \\
    Execute until the current \Ref{user cursor}{dialog:navigating:cursors}
    position is reached.
\item[{\bf Run to epilogue}] \mbox{} \\
    Execute until the current procedure's stack cleanup code.
\item[{\bf Run until address}] \mbox{} \\
    Execute until a given address is reached.
\item[{\bf Restart}] \mbox{} \\
    Restart the debuggee and stop at the main module body.
\item[{\bf Restart at startup}] \mbox{} \\
    Restart the debuggee and stop at the very beginning - the startup code.
\item[{\bf Restart at entry point}] \mbox{} \\
    Restart the debuggee and stop at the entry point.
\end{description}

\subsection{Run}
\label{dialog:executing:run}

The {\bf Run} command resumes execution of your program from
the current position at full speed, provided that no
\See{condition breaks}{}{dialog:breaks:condition} are installed and enabled.

Shortcut key: {\bf F5}

\subsection{Execution Trace}
\label{dialog:executing:trace}

The {\bf Execution trace} command executes the debuggee line by line
or instruction by instruction, moving the execution cursor
and updating all information windows after each step. This enables
you to watch control and data flow. Execution continues until you press {\bf Esc}
and may also be interrupted by a breakpoint or a break.

You may control whether code, for which no debug information is available,
is to be traced, by setting the \See{{\bf Never fall into disassembler} option}{}
{dialog:options}.

The \See{{\bf Delay in execution trace} option}{}{dialog:options} allows you to specify
a delay taken before execution of each line or instruction.

Shortcut key: {\bf Ctrl+F5}

\subsection{Step into}
\label{dialog:executing:stepinto}

The {\bf Step into} command executes the current line or instruction.
If it contains a procedure call, execution stops at the first line
or instruction of a called procedure.

The \See{{\bf Never fall into disassembler} option}{}{dialog:options} controls whether
procedures for which no debug infomation is available are traced
or executed in one step.

Shortcut key: {\bf F7}

\subsection{Step over}
\label{dialog:executing:stepover}

The {\bf Step over} command executes the current line or instruction.
If it contains any procedure calls, they are executed at once, unless
a breakpoint is encountered or a break is activated.

Shortcut keys: {\bf Ctrl+F7, Space}

\subsection{Step out}
\label{dialog:executing:stepout}

The {\bf Step out} command places a temporary breakpoint right after
the \verb'call' instruction which called the current procedure, and then
resumes execution. If no other breakpoints or breaks are encountered,
the debuggee will be executed until the current procedure finishes,
and then control will be returned back to the debugger. You may use
this command if you accidentally issued the \Ref{{\bf Step into}}
{dialog:executing:stepinto} command instead of
\Ref{{\bf Step over}}{dialog:executing:stepover}.

Shortcut key: {\bf Ctrl+F6}

\subsection{Run to cursor}
\label{dialog:executing:tocursor}

The {\bf Run to cursor} command installs a temporary breakpoint at
the current position of the user cursor, which has to be on an
\Ref{executable line}{start:general:executable},
and then resumes execution of the debuggee.
So execution will stop at that line, provided that no other breakpoints
or breaks are encountered, no exception is raised and the debuggee
is not terminated.

Shortcut key: {\bf F4}

\subsection{Run to epilogue}
\label{dialog:executing:toepilogue}

The {\bf Run to epilogue} command executes the debuggee until
the \See{epilogue}{}{start:general:epilogue} of the current procedure is reached.
This allows you to examine its local variables just before the procedure returns.

This command is available only for procedures with debug information.

\subsection{Run until address}
\label{dialog:executing:address}

The {\bf Run until address} command installs a temporary breakpoint
at an address you specify and then resumes execution of your
program.

{\bf Warning:} Do not use this command unless you know for sure
that there is an instruction starting at that address,
otherwise you will corrupt the code and execution will not stop.

\subsection{Restart}
\label{dialog:executing:restart}

The {\bf Restart} command reloads the debuggee and exectues it up to 
the beginning of the main module body. All
\Ref{breakpoints}{dialog:breaks}, \Ref{breaks}{dialog:breaks},
and \Ref{watches}{dialog:data:watches} are retained.

{\bf Note:} Your program may load DLLs at run-time, which will
be unavailable after restart. If there were any breakpoints
set in that DLLs, they will be disabled and a warning
message will be displayed.

Shortcut key: {\bf Ctrl+X}

\subsection{Restart at startup}
\label{dialog:executing:startup}

The {\bf Restart at startup} command is similar to the
\See{{\bf Restart}}{}{dialog:executing:restart} command
except that no debuggee instructions are executed after restart.
This allows you to debug DLL statrup code.

\subsection{Restart at entry point}
\label{dialog:executing:entry}

% !!! New in v1.1:

The {\bf Restart at entry point} command is similar to the
\See{{\bf Restart}}{}{dialog:executing:restart} command
except the debuggee is stopped at the EXE component's entry point, 
when all load-time DLLs are loaded and intialized.

\subsection{The Call Stack window}
\label{dialog:executing:callstack}

The {\bf Call Stack} window can be displayed by selecting the
correspondent item from the {\bf Code} menu. It contains the names
of the procedures which formed the call chain that led to the current
position in the code. The current procedure is the topmost in the window.

To have a certain procedure displayed in the {\bf Program} window,
double click its name, or select it using arrow keys and press {\bf Enter}.
% !!! New in version 1.1:
A mark will appear in front of that procedure name in the {\bf Call Stack} 
window, and the \See{{\bf Local variables} window}{}{dialog:data:locals}, 
if open, will display local variables an parameters of that procedure.

{\bf Note:} In some cases, the debugger may be unable to correctly trace back the
call chain, so the {\bf Call Stack} window may contain some garbage entries.
Turn the {\bf GENFRAME} compiler option ON to improve stack tracing accuracy.

\subsection{The Threads window}
\label{dialog:executing:threads}

The {\bf Threads} window displays information about all threads
started by the debuggee. It can be opened by selecting {\bf Threads}
from the {\bf Code} menu.

One of the threads is marked as current. When the debuggee is stopped,
you may switch to another thread by double-clicking it or selecting
with arrow keys and pressing {\bf Enter}.

% Write more !!!

\section{Using breakpoints and breaks}
\label{dialog:breaks}

A {\it breakpoint} is a mark in the debuggee's code which, once encountered
({\it hit}) during execution, causes it to stop and control to be trasferred
to the debugger. The source line or instruction on which the hit breakpoint is set is
{\it not} executed.

A {\it break} is a condition which is constantly checked during execution.
Once it evaluates to true, the debuggee is immediately stopped.

The major difference between breakpoints and breaks is that a breakpoint is
bound to a particular source line or instruction, while a break is activated
regardless of the current execution point. Using breakpoints, you may transfer
control to the debugger when a certain code is about to be executed.
Using breaks, you may force the debuggee to be stopped right after
a particular event, such as modification of a specific variable, happens in it.

To set a breakpoint, move the user cursor to the desired line or instruction,
and select an appropriate command from the {\bf Breaks} menu (or press a shortcut key).
Note that optimizations may cause strange effects during debugging, for instance,
some lines containing statements may be displayed as unexecutable.
See \ref{start:general} for more information.

{\bf Note:} Your program may load and unload DLLs at run-time.
If there were any breakpoints set in such a DLL, and your program unloads
it, the breakpoints are disabled and a warning message is displayed.
The same happens in the case of a restart.

For historical reasons, XD supports only seven types of breakpoints
which may not be combined together. This issue will be addressed
in future releases.

To set a break, select an appropriate command from the {\bf Breaks} menu
(or press a shortcut key). \See{Access breaks}{}{dialog:breaks:access}
may also be set from pop-up menus in some of the windows.

The commands related to breakpoints and breaks are collected in the {\bf Breaks}
menu:
\begin{description}
\item[{\bf Sticky breakpoint}] \mbox{} \\
    Set a sticky breakpoint at the current user cursor position.
\item[{\bf Breakpoint}] \mbox{} \\
    Set an automatically removable breakpoint.
\item[{\bf Delayed sticky breakpoint}] \mbox{} \\
    Set a sticky breakpoint with a countdown of hits.
\item[{\bf Delayed breakpoint}] \mbox{} \\
    Set a removable breakpoint with a countdown.
\item[{\bf Expression breakpoint}] \mbox{} \\
    Set a sticky breakpoint which triggers when a given expression is true.
\item[{\bf Pass counter}] \mbox{} \\
    Set a sticky breakpoint which only counts number of hits.
\item[{\bf Watchpoint}] \mbox{} \\
    Set a sticky breakpoint which automatically resumes execution.
\item[{\bf Access break}] \mbox{} \\
    Define an access break.
\item[{\bf Condition break}] \mbox{} \\
    Define a condition break.
\item[{\bf Disable}] \mbox{} \\
    Disable the breakpoint at the current user cursor position.
\item[{\bf Enable}] \mbox{} \\
    Enable a previously disabled breakpoint.
\item[{\bf Delete}] \mbox{} \\
    Remove the breakpoint at the current user cursor postiion.
\item[{\bf View all breaks}] \mbox{} \\
    Display the {\bf Breaks} window.
\item[{\bf Disable all breaks}] \mbox{} \\
    Disable all breaks and breakpoints.
\item[{\bf Enable all breaks}] \mbox{} \\
    Enable all breaks and breakpoints.
\item[{\bf Erase all breaks}] \mbox{} \\
    Remove all breaks and breakpoints.
\end{description}

\subsection{Sticky breakpoint}
\label{dialog:breaks:sticky}

The {\bf Sticky breakpoint} command sets a breakpoint
at the line or instrution pointed to by the user cursor.
It is called "sticky" because it is not removed when hit,
unlike a \Ref{simple breakpoint}{dialog:breaks:simple}.

The $\Diamond$ symbol is displayed in the mark column for
breakpoints in this type.

Shortcut key: {\bf F9}

\subsection{Breakpoint}
\label{dialog:breaks:simple}

The {\bf Breakpoint} command sets a simple breakpoint which is
automatically removed when it is hit.

The $\rhd$ symbol in the mark column indicates a simple breakpoint.

Shortcut key: {\bf F8}

\subsection{Delayed sticky breakpoints}
\label{dialog:breaks:delayedsticky}

The {\bf Delayed sticky breakpoint} command sets a
\Ref{sticky breakpoint}{dialog:breaks:sticky}
which stops the debuggee only after the specified number of hits.
This is very useful for debugging loops.

When you select this command, a dialog window is displayed,
prompting you for an initial countdown value. Type the number
of passes you wish the breakpoint to skip and press {\bf Enter}.

% !!! New in 1.1:

In a typical debug scenario a delayed breakpoint is used in conjunction with 
a \See{counter}{}{dialog:breaks:counter}: you first use a counter to 
determine how many times a particular line has been executed before crash,
then establish a delayed breakpoint at that line so that your program will
be stopped on the last iteration before crash, and then restart your program.
So if there is already a \See{counter}{}{dialog:breaks:counter}
established at the current line/address, the initial countdown
value in the dialog will be automatically set to the current value 
of the counter decremented by 1. 

When the countdown reaches zero, the breakpoint turns into a
regular \See{sticky breakpoint}{}{dialog:breaks:sticky}.

For a delayed sticky breakpoint, XD displays the remaining number of
passes (or the * symbol if it exceeds 9) and the $\Diamond$ symbol in the mark column.

Shortcut key: {\bf Ctrl+F9}

\subsection{Delayed breakpoint}
\label{dialog:breaks:delayed}

The {\bf Delayed breakpoint} command sets a delayed breakpoint
which is automatically removed after final hit.
See \ref{dialog:breaks:delayedsticky} for more infomation about
usage of delayed breakpoints.

A digit equal to the number of passes left (or the * symbol),
followed by the $\rhd$ symbol, is displayed in the mark column
for a delayed breakpoint.

Shortcut key: {\bf Ctrl+F8}

\subsection{Expression breakpoint}
\label{dialog:breaks:expression}

The {\bf Expression breakpoint} command installs a
\Ref{sticky breakpoint}{dialog:breaks:sticky}
and associates a boolean expression with it. The expression is evaluated each
time the breakpoint is hit, and execution is stopped only if the result is
\verb'TRUE'.

When the \Ref{user cursor}{dialog:navigating:cursors}
is positioned over a line containing an expression
breakpoint, the associated expression is shown in the {\bf Program}
window header.

A question mark (\verb'"?"') in the mark column indicates an expession
breakpoint.

Shortcut key: {\bf Alt+F9}

\subsection{Pass counter}
\label{dialog:breaks:counter}

The {\bf Pass counter} command establishes a
\Ref{sticky breakpoint}{dialog:breaks:sticky}
which, however, does not cause execution to stop but just counts the number of hits.
A pass counter initially has a value of zero which is incremented
each time the line or instruction associated with it is
executed.

Pass counters provide a method to determine how many times,
if at all, a particular line or instruction is encountered during
program execution.
This information may be used, for instance, when setting delayed
breakpoints.

When the \Ref{user cursor}{dialog:navigating:cursors} is positioned over
a pass counter, its value is shown in the {\bf Program} window header.
You may also use the \See{{\bf Breaks} window}{}{dialog:breaks:window}
to examine pass counter values.

For a pass counter, its current value followed by the "." symbol
is displayed in the mark column; values exceeding 9 are represented
with the "*" symbol.

Shortcut key: {\bf .}

\subsection{Watchpoint}
\label{dialog:breaks:watchpoint}

The {\bf Watchpoint} command sets a
\Ref{sticky breakpoint}{dialog:breaks:sticky},
upon encountering which execution is resumed automatically.
The effect is that all \Ref{information windows}{dialog:intro}
are updated when a watchpoint is passed.

This approach is in many cases more efficient than
\See{execution trace}{}{dialog:executing:trace}
in the sense that a program may be executed at full speed (for instance,
using the \See{{\bf Run} command}{}{dialog:executing:run}),
and information about its current state will be updated at and
only at the moments you specify.

The $\circ$ symbol in the mark column indicates a watchpoint.

\subsection{Access break}
\label{dialog:breaks:access}

The {\bf Access break} command allows you to have execution of your program
interrupted immediately after it accesses a certain block of memory.

{\bf Note:} Since access breaks are implemented using CPU debug registers,
they {\it do not slow down} execution of a debuggee, but also
have certain restrictions:
\begin{itemize}
\item Only blocks of 1, 2, or 4 bytes may be monitored
\item A block has to be aligned, i.e. its address must be a multiple of its length
\item Only write or read/write access may be detected
\item Only four access breaks may be enabled at a time
\end{itemize}

When you activate this command, the {\bf Access break}
dialog is displayed.
\ifcomment
\begin{figure}[h]
\begin{center} \ \drawing{accbrk}\ \end{center}
\end{figure}
\fi
Select the desired access type and memory block length,         % ??? More detailed style?
type an \See{address expression}{Chapter }{expr} in the {\bf Location} field,
and select the {\bf OK} pushbutton.

{\bf Note:} If there are already four enabled access breaks, the
new one will be created in disabled state and a warning message will be issued.

Shortcut key: {\bf Ins}

\subsection{Condition break}
\label{dialog:breaks:condition}

The {\bf Condition break} command allows you to specify a
boolean expression which will be constantly evaluated during
execution of your program until it yields \verb'TRUE', causing
execution to be stopped.

{\bf Note:} if there is at least one enabled condition break, your program
will be stopped after each CPU instruction, so its execution will slow
down dramatically. Avoid using condition breaks whenever possible.

\subsection{The Breaks window}
\label{dialog:breaks:window}

The {\bf Breaks} window contains a list of all breakpoints and breaks
currently set. To display it, press {\bf Ctrl+B} or select
{\bf View all breaks} from the {\bf Breaks} menu.

For a breakpoint, the name of the corresponding module
followed by the breakpoint mark and the source line is displayed.
For an access break, access type and address range are displayed.
Finally, for a conditional break, the associated expression is displayed.

Double-clicking or pressing {\bf Enter} on a breakpoint moves
the \Ref{user cursor}{dialog:navigating:cursors} to the
breakpoint position. You may also achieve this by selecting {\bf Go to}
from the pop-up menu.

Double-clicking or pressing {\bf Enter} on a break displays a
corrsponding dialog, allowing you to edit break parameters.
The {\bf Modify} command from the pop-up menu does the same.

\subsection{Managing breaks and breakpoints}
\label{dialog:breaks:manage}

A break or breakpoint may be {\it disabled} at any time.
A disabled breakpoint, once encountered, does not cause
the debuggee to stop executing. The associated counter, if any,
is not changed either. Marks corresponding to disabled breakpoints
are displayed in different color.

To disable a breakpoint, move the user cursor to the line
containing it or select it in the {\bf Breaks} window
and press {\bf "-"} or select {\bf Disable} from the {\bf Breaks} menu.
Press {\bf "+"} or select {\bf Enable} from the {\bf Breaks} menu to
re-enable a previously disabled breakpoint.

To delete a break or breakpoint, select it in the {\bf Program}
or {\bf Breaks} window and press {\bf Del} or select {\bf Delete}
from the {\bf Breaks} menu.

You may disable, enable, or delete all breakpoints and breaks at once
by selecting an appropriate item from the {\bf Breaks} menu.

\section{Examining your program data}
\label{dialog:data}

The {\bf Data} menu contains commands which open different kinds
of information windows displaying the current state of your program's data:
variables, stack, CPU registers, etc.:

\begin{description}
\item[{\bf Examine data}] \mbox{} \\
    Display and/or modify value of a given variable, designator, or register.
\item[{\bf Evaluate expression}] \mbox{} \\
    Evaluate an expression.
\item[{\bf Global variables}] \mbox{} \\
    Display the {\bf Global variables} window containing global variables of 
    the current component.
\item[{\bf Module variables}] \mbox{} \\
    Display the automatic {\bf Module variables} window containing 
    global variables of the current module.
\item[{\bf Local variables}] \mbox{} \\
    Display the {\bf Local variables} window containing local variables
    and parameters of a procedure 
\item[{\bf Memory dump}] \mbox{} \\
    Open a new {\bf Dump} window, displaying raw memory at a given
    address.
\item[{\bf Registers}] \mbox{} \\
    Display the {\bf Registers} window containing CPU registers.
\item[{\bf Float Registers}] \mbox{} \\
    Display the {\bf Float Registers} window containing FPU registers.
\item[{\bf Stack}] \mbox{} \\
    Display the {\bf Stack} window, which contains memory at the stack
    pointer of the current thread.
\item[{\bf Add watch}] \mbox{} \\
    Add an expression to the {\bf Watches} window.
\item[{\bf Delete watch}] \mbox{} \\
    Delete an item from the {\bf Watches} window.
\item[{\bf Show watch window}] \mbox{} \\
    Display the {\bf Watches} window.
\item[{\bf Del all watches}] \mbox{} \\
    Clears the {\bf Watches} window.
\end{description}

\subsection{Displaying and modifying values}
\label{dialog:data:display}

The way a value of a variable or a watch expression is displayed
depends on its type.

Values of elementary types immediately follow
the correspondent variables and expressions in list windows.
For variables of set and structured types (arrays and records),
except character arrays, only type names are shown in list windows.
Double-clicking or pressing {\bf Enter} on such a variable causes a
structured variable window to be displayed.
Each line of those windows contains a single array element or record field.

% !!! New in 1.1:
Select {\bf Show address} item from the pop-up menu of a structured variable window
to toggle display of variable addresses in window headers on or off. 

The \See{{\bf Use single structure window} option}{}{dialog:options}
controls whether a new window is opened to display a structured
variable element which, in turn, is of a structured or pointer type.

% !!! New in 1.1:
You may alter the display format of a particular element of data using
the {\bf Type} \Ref{pop-up submenu}{dialog:intro:menus}. You may change radix
for wholes, turn pointer dereference on/off, etc.

To modify a variable of a basic, range, or enumeration type
displayed in a list window, double click it or select it using arrow
keys and press {\bf Enter}. Type a new value in a displayed dialog and
press {\bf Enter} or press {\bf Esc} if you do not want to modify the current value.

Double-clicking or pressing {\bf Enter} on a pointer variable
or expression has the same effect as it would have on a pointed-to
object\footnote{provided that the pointer is not {\tt NIL}.}.
Use the {\bf Ctrl+Enter} key combination to modify value of a
pointer variable itself.

\subsection{Examine variable}
\label{dialog:data:variable}

The {\bf Examine variable} command displays a dialog prompting you for
a variable name. Type it and press {\bf Enter} (in fact, you may type
not just a variable name, but a \See{designator}{Chapter }{expr},
such as \verb'parr^[5].field').
Depending on the type of an object denoted by the designator, either
dialog containing the current value or structured variable view window
will be displayed. See \ref{dialog:data:display} for more information.

Shortcut key: {\bf ?}

\subsection{Evaluate expression}
\label{dialog:data:expression}

The {\bf Evaluate expression} command allows you to evaluate an
arbitrary \See{expression}{Chapter }{expr} using the current values
of program objects. Type it in a displayed dialog and press {\bf Enter}.

See also \ref{dialog:data:display}.

\subsection{The Global variables window}
\label{dialog:data:globals}

The {\bf Global variables} window displays global variables of the current
component, e.g. C \verb'extern' variables.
To open it, select {\bf Global variables} from the {\bf Data} menu.

See also \ref{dialog:data:display}.

\subsection{The Module variables window}
\label{dialog:data:module}

The automatic {\bf Module variables} window displays global variables of 
the module currently displayed in the {\bf Program} window.
To open it, select {\bf Module variables} from the {\bf Data} menu or
press {\bf Ctrl+V}.

To display global variables of a particular module in a separate window,
select {\bf Show variables} from the pop-up menu of the 
\Ref{{\bf Modules} window}{dialog:navigating:modules}.

See also \ref{dialog:data:display}.

\subsection{The Local variables window}
\label{dialog:data:locals}

Select {\bf Local variables} from the {\bf Data} menu to open the
{\bf Local variables} window. That window displays parameters and local
variables of the procedure currently selected in the
\Ref{{\bf Call Stack}}{dialog:executing:callstack} window.
Name of that procedure is displayed in the window header.

A horizontal line separates parameters from locals. An exclamation mark
is displayed in the first column for
\See{register variables}{}{start:general:regvars}.

If the selected procedure is nested, and the option {\bf DBGNESTEDPROC}
was set ON during complation of the module containing it, parameters and
local variables of enclosing prodcures are also displayed in the
{\bf Local variables} window. Horizontal lines with procedure names
are used as separators.

You may quickly traverse the \See{call stack}{}{dialog:executing:callstack}
when the {\bf Local variables} window is active. Press {\bf Ctrl+Down}
to display locals and parameters of the calling procedure, {\bf Ctrl+Up} ---
of the called procedure.

See also \ref{dialog:data:display}.

\subsection{Memory dump windows}
\label{dialog:data:dump}

You may use {\bf Memory dump} windows to examine your program's
code or data as raw memory. To open a dump window, press
{\bf Ctrl+D} or select {\bf Memory dump} from the {\bf Data} menu.
A dialog will display, prompting you for a starting address.
Type an address expression and press {\bf Enter}. % ???

If the current window contains a variable list or a structured variable
and you are opening a dump window, the address of the currenly selected
element will be substituted into the entry field.

Memory may be displayed in different formats. Select {\bf Type} from
the pop-up menu to switch between them.

To modify a dump element, double-click it, or select it using arrow keys
and press {\bf Enter}.

To change dump origin, click in the address column or select
{\bf Change dump origin} from the pop-up menu.

% !!! New in version 1.1:

You can save a memory dump to file in text or raw form by selecting 
{\bf Save dump to file} from the pop-up menu.

\subsection{The Registers window}
\label{dialog:data:registers}

The {\bf Registers} window displays names and values of the CPU registers.
To open it, press {\bf Ctrl+R} or select {\bf Registers} from the {\bf Data}
menu.

Each time execution of a debuggee stops, register values which have been
changed since previous stop are highlighted.

To modify a register value, double click it or select it with arrow keys
and press {\bf Enter}.

\subsection{The Float Registers window}
\label{dialog:data:float}

The {\bf Float Registers} window displays names and values of the FPU registers.
To open it, select {\bf Float Registers} from the {\bf Data} menu.

\ifcomment !!!
Each time execution of a debuggee stops, register values which have been
changed since previous stop are highlighted.
\fi

To modify a register value, double click it or select it with arrow keys
and press {\bf Enter}.

\subsection{The Stack window}
\label{dialog:data:stack}

The {\bf Stack} window displays stack of the current thread in raw form,
starting from the current stack pointer position. It may be opened by selecting
{\bf Stack} from the {\bf Data} menu.

You may change the format in which stack elements are displayed by
selecting {\bf Type} from the pop-up menu.

To change value of a stack element, double click it or select it with arrow keys
and press {\bf Enter}.

\subsection{The Watches window}
\label{dialog:data:watches}

The {\bf Watches} window contains arbitrary expressions along with their values.
The expressions are re-evaluated in the current execution context each time
the debuggee is stopped or program data is altered by the user.
If an expression contains variables which are undefined or are not visible
at the current execution point, an appropriate message is displayed instead
of expression value.

Expression results are displayed according to their types.
See \ref{dialog:data:display} for more information.

Each time execution of a debuggee stops, expressions which results have
changed since previous stop are highlighted.

To enter a new watch expression, select {\bf Add watch} from the {\bf Data}
menu or press {\bf Ctrl+Ins}. The {\bf Watches} window will be automatically
opened. To display it any time later, select {\bf Show watch window} from
the {\bf Data} menu.

To delete a watch expression, select it and press {\bf Del} or choose
{\bf Del watch} from the pop-up menu. To clear the {\bf Watches} window,
select {\bf Del all watches} from the {\bf Data} menu.

\section{Saving scripts}
\label{dialog:scripts}

% !!! New in 1.1:

When you find a suspicious place in your program, you may want to modify it
slightly in order to prove or disprove your hypothesis. Then you will have to
rebuild it and repeat the steps you took during the previous debugging
session. You could save some time on the last stage by using the {\bf Save script}
command from the {\bf File} menu before exitting the debugger. This command
creates a \See{control file}{Chapter }{batch} that, if invoked, would restore
all breakpoints, breaks, and watches currently set in the debugger.

\section{Setting XD options}
\label{dialog:options}

Selecting {\bf Options} from the {\bf File} menu causes an option sheet
to be displayed. These options are used to control various aspects of XD
behaviour in dialog mode.

\begin{description}
\item[{\bf Delay in execution trace}] \mbox{} \\
    The delay in milliseconds used in
    \See{{\bf Execution trace} mode}{}{dialog:executing:trace}.
\item[{\bf Exception raise on first chance}] \mbox{} \\
    If this option is enabled, the debugger displays a
    dialog immediately after an exception is raised in the program,
    allowing you to examine its state.
\item[{\bf Delay to select debuggee}] \mbox{} \\
    Specifies delay in milliseconds before the debuggee session
    is selected (see \ref{dialog:intro:sss}).
\item[{\bf Never fall into disassembler}] \mbox{} \\
    Check this box to disable automatic switching
    from the {\bf Source} mode to the {\bf Disassembly} mode
    during execution.
\item[{\bf Code highlight}] \mbox{} \\
    If enabled, XD displays \Ref{executable lines}{start:general:executable}
    in different color.
\item[{\bf Full disasm mode}] \mbox{} \\
    If enabled, XD tries to resolve addresses to publics and
    variable names in disassembly mode. Uncheck this box if you
    have a slow system, especially if you are running Windows 95.
\item[{\bf Dump in disassembler mode}] \mbox{} \\
    If enabled, bytes which constitute CPU instructions are displayed
    in {\bf Disassembly} and {\bf Mixed} modes.
% !!! New in version 1.1:
\item[{\bf Dereference pointers}] \mbox{} \\
    If enabled, the debugger displays next to pointer variables values of
    objects to which they refer (if the base type is simple), base type
    name (if the base type is named compound type), or base type sort
    (if the base type is anonymous compound type).
\item[{\bf Strip path from source name}] \mbox{} \\
    If this option is enabled, the debugger uses its redirection file
    to locate source files. This may be useful if you debug a program which
    was compiled in a different environment.
\item[{\bf Show all modules}] \mbox{} \\
    Defines whether the \See{{\bf Modules} window}{}{dialog:navigating:modules}
    lists all modules or only those for which debug information is available.
% !!! What about Show all Components?
\item[{\bf Auto detect actual type}] \mbox{} \\
% !!! New in version 1.1:
    Enables detection of the actual dynamic type of an Oberon-2 pointer 
    or a Java object. If this option is disabled, actual type of a 
    particular entity can still be determined using a
    \Ref{pop-up menu}{dialog:intro:menus}.
\item[{\bf Use single structure window}] \mbox{} \\
    Controls whether a new window is opened when you select an array element
    or a record field of a structured or pointer type.
\item[{\bf Warning bell}] \mbox{} \\
    Enables or disables speaker beeping when an error message is displayed.
\item[{\bf Save debug session layout}] \mbox{} \\
    Enabling this option causes information about debugger options,
    window layout and colors, and watch expressions to be automatically 
    stored in a \verb'.ses' file upon debugging session termination.
\item[{\bf Use keyboard layout}] \mbox{} \\
    When enabled, forces XD to load keyboard shotcuts upon startup 
    from the file specifed in the entry field; otherwise, the default 
    layout is used (see \ref{dialog:options:shortcuts}).
\end{description}

Use the {\bf Save config} command from the {\bf File} menu to store the
current option settings, window layout, and colors as the default.

\subsection{Changing window colors}
\label{dialog:options:colors}

To modify XD color schemes, select {\bf Palette} from the {\bf File} menu.
A window containing a list of all window types will appear.
Selecting a window type will cause its color sheet to be displayed.
A color sheet consists of two columns, the left containing names of window
areas, and the right containing current colors for each area.

The following keys may be used in a color sheet:

\KeyListBegin{10 cm}
\KeyDef{Up/Down}          {move the cursor one line up/down}
\KeyDef{Left/Right}       {change foreground color}
\KeyDef{Ctrl+Left/Right}  {change background color}
\KeyDef{Esc}              {close the color sheet}
\KeyListEnd

If you have the \See{{\bf Save debug session layout} option}{}{dialog:options}
switched ON, the new color scheme will be stored in a
\Ref{session file}{dialog:sysfiles}
upon debugging session termination. Otherwise, the changes you made
will be lost unless you explicitly save the current configuration
using the {\bf Save config} command from the {\bf File} menu.

\subsection{Modifying keyboard shortcuts}
\label{dialog:options:shortcuts}

If the option \See{{\bf Use keyboard layout}}{}{dialog:options} is enabled,
upon startup the debugger attempts to load keyboard shortcuts from the
file specified in the respective entry field. If that name does not contain
directories, the file is sought via redirections, then is the current directory,
and then in the directory where XD executable resides.

Default shortcuts are hard-wired into the debugger, so in order to modify
them, you need to save the default layout to file using the
{\bf Save keyboard layout} command from the {\bf File} menu. A
keyboard shortcut file is a plain text file that looks similar to
the following:

\begin{verbatim}
    [ Actions and keys ]
    Main pulldown                    = F10
    Main window                      = Ctrl-T
    Menu: File                       = Alt-F
    Load program                     = F3
    Refresh screen                   = Ctrl-E
    Show output                      = /
    Show debuggee                    = \
    Palette                          = 
    Options                          = 
       .  .  .
\end{verbatim}

Edit the layout file in any text editor, then enable the 
\See{{\bf Use keyboard layout}}{}{dialog:options} and specify the name of
that file in the respective entry field.
described above.

\subsection{Modifying frames}

If the default window frames do not suite you, edit the \verb'FrameImageSingle',
\verb'FrameImageDouble', and \verb'FrameImageMove' items in the \verb'[ Options ]'
section of the debugger configuration file. 
