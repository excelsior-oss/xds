\chapter{The HIS utility}
\label{his}

\XDS{} runtime support contains routines that in case of abnormal program
termination may trace back the stack of the thread that is in error and
output the call chain into a file called \verb'errinfo.$$$'. To enable this,
the option {\bf GENHISTORY} has to be turned ON during compilation of the main
module of the program. % !!! extref to XC User's Guide.

The HIS utility takes the stack trace file as input and prints execution history
of the crashed thread in terms of line numbers and procedure names (provided that
the respective executable components contain debug information):

{\footnotesize
\begin{verbatim}
    XDS History formatter, Version 2.0
    (c) 1996-2000 Excelsior
    
    F:\xds\samples\modula\hisdemo.exe
    -------------------------------------------------------------------------------
    #RTS: unhandled exception #6: zero or negative divisor
    -------------------------------------------------------------------------------
    Source file                     LINE  OFFSET  PROCEDURE        COMPONENT
    -------------------------------------------------------------------------------
    hisdemo.mod                        5  00000F  Div              hisdemo.exe
    hisdemo.mod                       11  000037  Try              hisdemo.exe
    hisdemo.mod                       15  000061  main             hisdemo.exe
\end{verbatim}
} % \footnotesize

In the above example, division by zero has been detected at line 5 in the
procedure \verb'Div' that was called by the procedure \verb'Try' at line 11, and the
procedure \verb'Try', in turn, was called from the main module body at line 15.
All these procedures are defined in file \verb'hisdemo.mod' and their code reside
in the executable file \verb'hisdemo.exe'.

By default, HIS prints to the standard output. Use \verb'-l=file' 
option to redirect it to \verb'file'.

The \verb'-f' option tells HIS to output full names of source files.

\pagebreak
HIS is also able to facilitate quick reproduction of the erroneous situation in XD.
Launching HIS with the \verb'-b=file' option creates a
\See{debugger control file}{Chapter }{batch}
that would stop the program immediately before execution of the instruction that
causes crash:

{\footnotesize
\begin{verbatim}
    ; 
    ; XDS History formatter, Version 2.0
    ; (c) 1996-2000 Excelsior
    ; 
    ; XDS Debugger control file
    
    ; 1. Load the program
    
      LOAD F:\xds\samples\modula\hisdemo.exe  
    
    ; 2. Establish breakpoint-counter
    
      MODULE hisdemo$
      BREAK Counter, LINE, hisdemo, 5, .
    
    ; 3. Initiate program execution at the first time
    
      START
    
    ; 4. When program has finished, establish delayed sticky breakpoint
    
      MODULE hisdemo$
      BREAK Break, LINE, hisdemo, 5, Switch_To_Dialog, @PASS(Counter)-1
    
    ; 5. Delete auxiliary breakpoint
    
      DEL Counter
    
    ; 6. Initiate program execution for the second time
    
      RESTART
      START
    
      QUIT
    
    ; 7. Upon the breakpoint hit, switch to the dialog mode
    
    Switch_To_Dialog
      DIALOG #RTS: unhandled exception #6: zero or negative divisor
      STOP
\end{verbatim}
} % \footnotesize

You may use the \verb'-c' option if you do not want comments to appear in the 
control file.

The \verb'-d' option forces HIS to immediately launch the debugger with the created
control file. If the \verb'-b' option was not specified, a temporary control file 
is created.

The \verb'-s' option disables trace of the control file execution in the debugger.


