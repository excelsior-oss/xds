\chapter{Managing libraries}
\label{xlib}

Libraries are files of a special format that are used to collect
object modules or information required to tune import from DLLs
(so called "import libraries").

The XDS Library Manager (also referred to as XLIB in the documentation),
is intended for library creation and management.

\section{Starting XLIB}
\label{xlib:start}

To invoke XLIB, enter \verb'"xlib"' with necessary arguments
at the command prompt. An argument may be either an option,
a file name, a command, or a response file name. The
command line syntax varies among \See{operation modes}{}{xlib:modes}.

Being invoked without additional arguments,
XLIB outputs a brief help message.

\subsection{Options}

Options may be placed on the command line in arbitrary order.
An option consists of the slash ("\verb'/'") character immediately
followed by an option name. Some options also have string values
delimited with the "\verb'='" character.

The following table lists XLIB options along with their meanings and
default values.

\begin{tabular}{ll}
\bf Option   & \bf Meaning \\
\hline
\tt /IMPLIB         & \See{Create an import library}{}{xlib:modes:import} \\
\tt /EDF            & \See{Create an export definition file}{}{xlib:modes:export} \\
\tt /LIST           & \See{List contents of a given library}{}{xlib:modes:list} \\
\tt /HELP           & Output a brief help text and terminate \\
\tt /NOLOGO         & Suppress logo \\
\tt /NOBAK          & Do not backup files before overwriting them \\
\tt /NEW            & Always create a new library \\
\tt /PREFIX=String  & \See{Set prefix of imported names}{}{xlib:modes:import} \\
\tt /POSTFIX=String & \See{Set postfix of imported names}{}{xlib:modes:import} \\
\tt /USEORD         & Use ordinal numbers (see \ref{xlib:modes:import} and \ref{xlib:modes:export})
\end{tabular}

If XLIB encounters an unknown option, or an option that has no
effect in a given operation mode, it produces a warning message.

\subsection{File names}

File names appearing on XLIB command line are regular file
names. Each name may be prefixed with directory path.
If an extension is omitted, defaults are used:

\begin{tabular}{ll}
\bf Type         & \bf Extension \\
\hline
Library                & \tt .lib \\
Object module          & \tt .obj \\
DLL                    & \tt .dll \\
Export definition file & \tt .edf \\
Import definition file & \tt .idf
\end{tabular}

\subsection{Response files}

A response file argument is the "\verb'@'" character immediately
followed by a response file name.
A response file is a text file that contains a number
of XLIB arguments, one on each line. Any type of argument is allowed
except another response file.

{\bf Note:} The options /IMPLIB, /LIST, /EDF, /HELP, and /NOLOGO
have no effect when used in a response file.

\subsection{Commands}
\label{xlib:start:commands}

A command is a non-empty set of command symbols
("\verb'+'","\verb'-'", and "\verb'*'")
followed by the name of the object module or file which is the
subject of the command. Relational order of command symbols has no meaning.

Note that commands are valid for the
\See{library management operation mode}{}{xlib:modes:management} only.

\section{Operation modes}
\label{xlib:modes}

XLIB has four operation modes, switched by
/IMPLIB, /LIST, and /EDF options as follows:

\begin{tabular}{ll}
\bf Option & Mode \\
\hline
default  & Library management \\
/IMPLIB  & Import library creation \\
/LIST    & Library contents listing \\
/EDF     & Export info extraction
\end{tabular}

Details of operation modes are given in the following sections.

\subsection{Library management}
\label{xlib:modes:management}

\begin{verbatim}
    xlib { option } libname[.lib] { command }
    command = { +|-|* }module[.obj] | @ResponceFileName
\end{verbatim}

In this mode XLIB acts as a proper library manager. The first
argument which does not represent an option setting is treated as
the name of the library file. If an extension is omitted, the
default "\verb'.lib'" is added. If the /NEW option is not set,
XLIB supposes that the given library file already exists and is
valid, otherwise it produces a warning and creates a new library file.

Arguments following a library name are treated as
\See{commands}{}{xlib:start:commands}.
Each command represents a set of actions and
a name of an object module or library.

Valid actions are add, remove, and extract, denoted by the
"\verb'+'", "\verb'-'" and "\verb'*'" characters respectively.
When more than one action is specified, XLIB performs them in the following order:

\begin{enumerate}
\item extract
\item remove
\item add
\end{enumerate}

For instance, to replace the module "\verb'a.obj'"
you may use either "\verb'-+a'" or
"\verb'+-a'" command.

If all commands were successfully executed, XLIB checks the resulting
library for duplicate names, producing warnings if necessary.

Finally, the new library is written to disk. The original library file,
if existed before XLIB execution, is saved by changing its
extension to \verb'.BAK', unless the /NOBAK
option is specified.

\subsection{Import library creation}
\label{xlib:modes:import}

\begin{verbatim}
    xlib /IMPLIB { option } libname[.lib] { File }
    File = filename([.dll]|.exe|.idf) | @ResponceFileName
\end{verbatim}

In this mode XLIB produces an \See{{\em import library}}{}{dll:using:load-time}.
Libraries of this kind contain information
which is used by linkers to tune up import sections in
executable files (see \ref{dll:using:load-time}).
Import libraries are commonly used instead of definition files.

To switch XLIB to import librarian mode, specify the /IMPLIB
option on the command line.

In this mode, XLIB treats the first given argument that does not
represent an option setting as the name of the output library file.
If an extension is omitted, the default \verb'.lib' is added. XLIB
always creates a new library file. If the output file
does already exist, XLIB saves it by changing its extension
to "\verb'.bak'", unless the /NOBAK option is set.

Other arguments that do not represent option settings are treated
as names of source files. XLIB may retrieve information required
to create an import library from two kinds of sources, namely from
an executable (EXE or DLL) or from an
\See{import definition file}{}{xlib:modes:import:IDF}.
The type of source is determined by file extension,
where "\verb'.exe'" stands for an executable,
"\verb'.dll'" stands for a dynamically linked library,
and "\verb'.idf'" stands for an import definition file.
If an extension is omitted, "\verb'.dll'" is added by default.

See \ref{xlib:formats} for details of supported formats.

In general, an import library sets correspondence between
entry names (e.g. names under which entries may be referenced
in other executable modules) and pairs consisting of a DLL name
and an internal name or ordinal number of an exported entry.
It may be necessary to mangle entry names to fit naming
convention of a programming system. XLIB provides a simple
mechanism for mangling. It uses value of
the /PREFIX option as a prefix and value of the
/POSTFIX option as a postfix for all entry names it produces.

For example, being invoked with

\verb'    /PREFIX=_IMP_ /POSTFIX=_EXP_'

XLIB will mangle the internal name "\verb'OneEntry'" to
the entry name "\verb'_IMP_OneEntry_EXP_'".

By default, XLIB creates import libraries which use import by name.
The /USEORD option forces it to use import by ordinal.
See \ref{dll:overview} and \ref{dll:using:load-time} for more
information.

After you have created an import library, you may use XLIB on it to
add object modules which you want to be available to a linker when you
link with that import library.

\subsubsection{Import definition files}
\label{xlib:modes:import:IDF}

Below is the description of the import definition files format.

\begin{verbatim}
    File = Import ";" { Import ";" }
    Import = FROM Module IMPORT Entry { "," Entry }
    Entry = InternalName [ AS EntryName ]
\end{verbatim}

\verb'Module' is a name of an executable module (DLL or EXE).
\verb'InternalName' is the entry name as it appears within that module.
Finally, \verb'EntryName' is the entry name as it is to be known to
other modules. If the \verb'AS EntryName' clause is omitted, the entry is
exported under its internal name.

Lines starting with "\verb'%'" are treated as comments and ignored.

Example of an import definition file:

\begin{verbatim}
%
%  Sample Import Definition File
%  Copyright (c) 1997 XDS Ltd.
%
FROM KERNEL32.DLL IMPORT
  CreateFileA
  ,WriteFile
  ,ReadFile
  ,DeleteFileA
  ,MoveFileA
  ,SetFilePointer
  ,SetSystemTime
  ,RtlFillMemory AS FillMemory
  ;
FROM USER32.DLL IMPORT
  ShowWindow
  ,ShowWindowAsync
  ,TileWindows
  ;
% end of the sample import definition file
\end{verbatim}

\subsection{Library contents listing}
\label{xlib:modes:list}

\verb'    xlib /LIST { option } libname[".lib"]'

In this mode XLIB outputs a library contents listing.

To switch XLIB to the library contents mode, specify the /LIST option
on the command line.

In this mode XLIB treats the first given argument as the input library
file name. If extension is omitted, the default "\verb'.lib'" will be
added. The library must exist and must be in correct format,
otherwise XLIB will produce an error report.

Other arguments that are not option settings are ignored.

Library contents are printed as follows:

\begin{verbatim}
    Library "libxds.lib"
      MODULE src\isoimp\chancons.mod
        ChanConsts_BEGIN
      MODULE src\isoimp\charclas.mod
        CharClass_IsControl
        CharClass_IsLetter
        CharClass_IsLower
        CharClass_IsNumeric
        CharClass_IsUpper
        CharClass_IsWhiteSpace
        CharClass_BEGIN
     .  .  .
\end{verbatim}

\subsection{Export info extraction}
\label{xlib:modes:export}

\verb'    xlib /EDF { option } edfname[.edf] dllname[.dll]'

If this mode, XLIB reads export information from the given DLL and
creates an \See{export definition file}{}{link:EDF} (EDF),
which may then be edited and used when the DLL is rebuilt
(See \ref{dll:create:export}).

If the option /USEORD is specified, the output EDF will contain
\See{ordinal numbers}{}{xlib:modes:import}.

The first non-option argument is treated as the output EDF name.
If no extension is specified, "\verb'.edf'" is assumed.
The second argument must be a DLL name.
If an extension is omitted, the default "\verb'.dll'" is assumed.

\section{Supported formats}
\label{xlib:formats}

Here the input and output formats currently supported by XLIB are
discussed.

\subsection{Output library format}

Currently XLIB produces output libraries in OMF format only.
This format is compatible with most of available linkers. The
important exception are Microsoft linkers which expect
libraries to be in COFF format.

COFF format will be supported later.

\subsection{Input object file format}

Currently XLIB expects all input object files to be in OMF
format. Many modern compilers, produce object files in this format.
The important exception are MSVC (v2.0 and higher) compilers
by Microsoft which produce output in COFF format.
XDS compilers may produce object files in both formats
(see the \OERef{OBJFMT} option).

COFF format will be supported later.

\subsection{Input executable format}

XLIB may retrieve information for import libraries
from executables (either DLL or EXE) which are in the PE (Portable
Executable) format, used for 32-bit executables in Windows.
No other formats are planned to be supported.

