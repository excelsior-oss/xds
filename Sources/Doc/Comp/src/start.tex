% !!! xdsuser -> xcwork xcsamp

\chapter{Getting started}\label{start}

In this and following chapters we assume that
\xds{} is properly installed and configured
(See Chapter \ref{config});
the default file extensions are used.

Your \XDS{} package contains a script file, \XC{}{\tt work},
which may be used to prepare a working directory.
For more information, consult your {\tt readme.1st} file from
the \XDS{} on-line documentation.

\section{Using the Modula-2 compiler}

In  the  working  directory,  use  a text editor to create a file
called {\bf hello.mod}, containing the following text:

\begin{verbatim}
    MODULE hello;

    IMPORT InOut;

    BEGIN
     InOut.WriteString("Hello World");
     InOut.WriteLn;
    END hello.
\end{verbatim}

Type

\verb'    '{\tt \XC{}}\verb' hello.mod'

at the command prompt. \xc{} will know that the \mt{} compiler
should be invoked for the source file with the extension {\bf.mod}.
The compiler heading line will appear:

\verb'    XDS Modula-2 '{\it version}\verb' ['{\it code generator}\verb'] "hello.mod"'

showing which compiler has been invoked
(including its version number), which code generator
is being used (in square brackets) and what is its version,
and finally the name of the source file \xc{} has been
asked to compile.

Assuming that you have correctly typed the source file,
the compiler will then display something like

\verb'    no errors, no warnings, lines   15, time  1.09'

showing the number of errors, the number of source lines
and the compilation time.

{\bf Note:} The \XDS{} compiler reports are user configurable.
If the lines similar to the above do not appear, check that the \OERef{DECOR}
equation value contains letters `C' (compiler heading) and `R' (report).

\section{Using the Oberon-2 compiler}

In our bilingual system the \mt{} source code just shown is also
perfectly valid as the \ot{} code. \xds{} allows you to
use \mt{} libraries when programming in \ot{} (in our case the
{\tt InOut} module).

As in \mt{}, this source code in \ot{} constitutes
a {\em top-level module} or {\em program module},
but in \ot{}, there is no syntactic distinction between
a top-level module and any other module.
The \ot{} compiler must be specifically told that this is
a top-level module by using the option \OERef{MAIN}.

Copy the source file to the file {\bf hello.ob2} and type:

\verb'    '\XC{}\verb' hello.ob2 +MAIN'

The same sequence of reports will occur as that of the \mt{} compiler,
but the \ot{} compiler will also report whether a new symbol
file was generated or not.
It is also possible to override the default source file extension using
\OERef{M2} and \OERef{O2} options:

\verb'    '\XC{}\verb' hello.mod +O2 +MAIN'

In this case, the \ot{} compiler will be invoked regardless
of the file extension.

\section{Error reporting}

If either compiler detects an error in your code, an error description
will be displayed. In most cases a copy of the source line will also be
shown with a dollar sign \verb|"$"| placed directly before the point at
which the error occurred. The format in which \xds{} reports errors is
user configurable (See \ref{opt:errfmt}), by default it includes a
file name, a position (line and column numbers) at which the error occurred,
an error type indicator, which can be [E]rror, [W]arning or [F]ault,
an error number, and an error message.

\Example
{\samepage
\begin{verbatim}
* [bf.mod 26.03 W310]
* infinite loop
  $LOOP
\end{verbatim}
} % samepage
\pagebreak[1]

\section{Building a program}\label{start:build}
\index{running a program}

\ifgencode        %----------------------

To have you program automatically linked, invoke the compiler in the
\See{MAKE mode}{}{xc:modes:make}:

\verb'    '\xc\verb' =m hello.mod'

In this mode, the compiler processes all modules which are imported by
the module specified on the command line, compiling them if necessary.
Then, if the specified module was a program module, the linker is invoked.

However, if your program consists of several modules, we recommend to
write a \See{project file}{}{xc:project}. In the simplest case, it consists
of a single line specifying a name of a main module:

\verb'    !module hello.mod'

but it may also contain various \See{option settings}{}{options}.
The following invocation

\verb'    '{\tt \xc{}}\verb' =p hello.prj'

will compile modules constituting the project (if required) and
then execute the linker.

Here is a more complex project file:
\begin{verbatim}
% debug ON
-gendebug+
-genhistory+
-lineno+
% optimize for Pentium
-cpu = pentium
% response file name
-mkfname = wlink
-mkfext  = lnk
% specify an alternate template file
-template = wlink.tem
% linker command line
-link = "wlink @%s",mkfname#mkfext;
% main module of the program
!module main.mod
% additional library
!module clib3s.lib
\end{verbatim}

After successful compilation of the whole project the compiler
creates a linker response file using the specified
\See{template file}{}{xc:template} and then executes a command line
specified by the \OERef{LINK} equation.

\fi % end ifgencode ----------------------

\ifgenc %----------------------

 After compilation of {\tt hello} program you can invoke your C compiler.
 \index{C compiler}
 It is necessary to specify the paths to header files and library
 (use the name of appropriate library from you package).
 Consult your C compiler manual for syntax of the compiler command line.
 \begin{verbatim}
   cc hello.c -Ic:\xds\include c:\xds\lib\libxds.lib
 \end{verbatim}
 Type
 \begin{verbatim}
   hello
 \end{verbatim}
 to run your program.

 The {\tt xdsuser} script creates the {\tt xcc} script that can be used
 to compile and link a simple program. Type
 \begin{verbatim}
   xcc hello.c
 \end{verbatim}
If your project contains more than one module, we recommend to
write a project file (See \ref{xc:project}) and use appropriate
template file (See \ref{xc:template}).
The following project file contains all necessary settings:
\begin{verbatim}
% debug ON
-gendebug+
-lineno+
% specify template file
-template = xds.tem
% specify a name of a makefile
-mkfname = tmp
-mkfext  = mkf
% force generation of the makefile
-makefile+
% call the make
-link = "make -f %s",mkfname#mkfext;
% main module of the program
!module hello.mod
\end{verbatim}
It specifies the template file to use ({\tt xds.tem}), the name
of the makefile ({\tt tmp.mkf}) and the make command line.

After successful compilation of the whole project the compiler
creates the makefile and then executes the command line,
specied by the {\bf LINK} equation.
The {\tt xds.tem} template file defines a template for a makefile.
The following invocation

\verb'    '\xc{}\verb' hello.prj =p'

will compile modules constituting the project (if required) and
then execute the make. See \ref{xc:template} for the full description
of template files.
See also the project files, generated by the {\tt xdsuser} script.

\fi %----------------------

\section{Debugging a program}\label{start:debug}
\index{debugging a program}

\ifgencode

\XDS{} compilers generate debug information in the
\iflinux stabs \else CodeView or HLL4 \fi format and
allow you to use any debugger compatible with that format \iflinux (for
example, GDB)\else or the XD debugger included
in your \XDS{} package\fi.  However,
the {\em postmorten history}\index{history}\index{postmorten history}
feature of \XDS{} run-time support may be used in many cases instead of
debugger. To enable this feature, the option \OERef{LINENO}
should be set on for all modules in the
program and the option \OERef{GENHISTORY}
for the main module of the program; the program also has to be linked
with debug info included\ifxlink{} (See \ref{link:inout:gendebug})\fi.
If your program was built with the above settings, the run-time system
dumps the stack of procedure calls on an abnormal termination
into a file called \verb'errinfo.$$$'. % What about Linux???
The HIS utility reads that file and outputs the call stack in % ??? name
terms of procedure names and line numbers using the debug info
from your program's executable file.

\Example
\begin{verbatim}
MODULE test;

PROCEDURE Div(a,b: INTEGER): INTEGER;
BEGIN
  RETURN a DIV b
END Div;

PROCEDURE Try;
  VAR res: INTEGER;
BEGIN
  res:=Div(1,0);
END Try;

BEGIN
  Try;
END test.
\end{verbatim}

\pagebreak[1]
{\samepage
When this program is running, an exception is raised and the
run-time system stores the exception location and a stack of
procedure calls in a file \verb'errinfo.$$$' and displays the
following message:

\begin{verbatim}
#RTS: unhandled exception #6: zero or negative divisor

File errinfo.$$$ created.
\end{verbatim}
} % \samepage
\pagebreak[1]

The \verb'errinfo.$$$' is not human readable. The HIS utility,
once invoked, reads it along with the debug information from
your program executable and outputs the call stack in a more
usable form:

\pagebreak[1]
{\samepage
\begin{verbatim}
#RTS: unhandled exception #6: zero or negative divisor
------------------------------------------------------------
Source file                     LINE  OFFSET  PROCEDURE
------------------------------------------------------------
test.mod                           5  00000F  Div
test.mod                          11  000037  Try
test.mod                          15  000061  main
\end{verbatim}
} % \samepage
\pagebreak[1]

The exception was raised in line 5 of {\tt test.mod},
the {\tt Div} procedure was called from line 11,
while the {\tt Try} procedure was called from line 15
(module body).

{\bf Note:} In some cases, the history may be inaccurate.
See \ref{rts:history} for further details.

\fi % ifgencode --------------------------

\ifgenc % debugging

\XDS{} allows one to use any standard debugger. However, the {\em
postmorten history}\index{history}\index{postmorten history} feature of
\XDS{} run-time support may be used in may cases instead of debugger.
To enable this feature the option \OERef{GENDEBUG}
should be set for all modules in the program;
the debug mode should also be set for a C compiler and linker.

If your program is compiled in this mode, the run-time system will
print a stack of procedure calls (a file name and a line number) on
abnormal termination of your program.

\Example
\begin{verbatim}
MODULE test;

PROCEDURE Div(a,b: INTEGER): INTEGER;
BEGIN
  RETURN a DIV b
END Div;

PROCEDURE Try;
  VAR res: INTEGER;
BEGIN
  res:=Div(1,0);
END Try;

BEGIN
  Try;
END test.
\end{verbatim}
When this program is running, an exception is raised and the
run-time system prints the exception location and a stack of
procedure calls. If the option \OERef{LINENO} is ON, all information
will be reported in terms of original (\ot{}/\mt{}) source files:
\begin{verbatim}
#RTS: No exception handler #6: zero or negative divisor.
test.mod              5
test.mod             11
test.mod             15
\end{verbatim}
The exception was raised in line 6 of {\tt test.mod},
the {\tt Div} procedure was called from line 12,
while the {\tt Try} procedure was called from line 16
(module body).
If the option \OERef{LINENO} is OFF, all information will be reported in
terms of generated C files:
\begin{verbatim}
#RTS: No exception handler #6: zero or negative divisor.
test.c               17
test.c               27
test.c               36
\end{verbatim}
See \ref{maptoc:opt:gen} for additional details.

\fi % end ifgenc (debugging) ----------------------


