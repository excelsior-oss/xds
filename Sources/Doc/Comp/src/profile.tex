\chapter{Analyzing your program profile}
\label{profiler}

Your XDS package contains a set of utilities that may help you
to better understand and improve the run-time performance of your
program. It emphasises the pieces of code which consume most of
the CPU time, and, hence, are the first candidates for redesign.

\section{Overview}
\label{profiler:overview}

The process of profiling takes two steps - execution tracing
and visualisation. During execution tracing, your program
is executed in debugging mode under control
of the tracing utility. That utility interrupts the program
at regular time intervals and records instruction pointer (EIP)
values (snapshots), storing them in a {\it trace file}.
A visualization utility reads that file and displays distribution
of snapshots among your program components, modules, procedures,
and source lines, using debug information from the program executable.

\section{Execution tracing}
\label{profiler:tracing}

Execution tracing is performed by the XPROF utility. To invoke it,
issue the following command:

\verb'    XPROF { ("/" | "-") option } program [ arguments ]'

where \verb'program' is a name of a profiled executable and
\verb'arguments' are its optional command line arguments ({\it not}
XPROF arguments).

Upon termination, XPROF creates a trace file in the current directory.
Its name is built by appending the extension \verb'.XPT' to the executable name.

Available options are:

\ifonline
\begin{tabular}{ll}
\else
\begin{tabular}{lp{11 cm}}
\fi
\tt /R=nnn & Set snapshot interval to {\tt nnn} ms
             Default is 55, minimum 32. \\
\tt /A     & Append trace data to the existent .XPT file.
\end{tabular}

The \verb'/A' option allows you to collect more snapshots,
improving profile accuracy.

\section{Visualisation}

The visualisation utilities, XPDUMP and XPVIEW, read the collected
trace data and collate it with the debug information bound to
the executable. The results of that analysys are then displayed.

\subsection{XPDUMP}

XPDUMP is a command line utility. Given a name of a trace file,
it prints to the standard output a mutlilevel list of components,
modules, procedures, and source lines of the traced program.

XPDUMP command line syntax:

\verb'XPDUMP { ( "-" | "/" ) option } tracefile'

XPDUMP options:

\begin{tabular}{ll}
\bf Option       & \bf Description \\
\hline
\tt /O=order     &  Sort order: by ({\tt N})ame or ({\tt P})ercent \\
\tt /L=level     &  Sensitivity level in percents               \\
\tt /P=precision &  Precision (number of decimal positions)     \\
\tt /R           &  Sort in reverse order                       \\
\tt /G           &  Display graph bar                           \\
\tt /C           &  Display components only                     \\
\tt /M           &  Display modules only                        \\
\tt /S           &  Display source                              \\
\tt /F           &  Show full path for modules
\end{tabular}

The default is \verb'/O=P /L=5 /P=1'.

Here is a sample XPDUMP output:

\ifonline\else{\footnotesize\fi
\begin{verbatim}
-----------------------------------------------------------------------------
Snapshots: 200
-----------------------------------------------------------------------------
 90.4    90 ####################################.... SAMPLES\BENCH\dry.mod
 37.7    34 ###############.........................     Proc0
   20    18 ########................................     Proc1
 13.3    12 #####...................................     Proc8
  8.3   7.5 ###.....................................     Func2
  7.7     7 ###.....................................     Proc7
\end{verbatim}
\ifonline\else}\fi % \footnotesize

"Snapshots" is a total number of EIP values recorded in the trace file.

\havetowrite about components !!!

For each module, the first column contains ratio between the number of
snapshots belonging to the module and the number of snapshots belonging to
all program modules. The second column contains  ratio between the number of
snapshots belonging to the module and the total number of snapshots.

For each procedure, the first column contains ratio between the number of
snapshots belonging to the procedure and the number of snapshots belonging to
the module which contains that procedure. The second column contains ratio
between the number of snapshots belonging to the procedure and the total
number of snapshots.

For each source line, the first column contains ratio between the number of
snapshots belonging to the line and the number of snapshots belonging to
the module which contains that line. The second column contains ratio
between the number of snapshots belonging to the line and the total
number of snapshots.

\subsection{Profile viewer}

XDS Profile Viewer is a GUI program which allows you to
interactively browse your program profile.
It can be invoked with the following command:

\verb'    XPVIEW [ tracefile ]'

where \verb'tracefile' is the name of the trace file which you want to display.
If you did not specify it on the command line, press {\bf F3} or select
{\bf Open} from the {\bf File} menu to load an .XPT file using a standard file dialog
once XPVIEW is up and running.

XPVIEW displays profile information in four windows.
The first window contains a list or your program components
(EXE and DLLs). The second window contains the list of the source
modules which constitute the currently selected component,
the third --- the list of procedures belonging to the currently
selected module, and the fourth --- the source code of the
currently selected procedure.

Snapshots percentage in numeric and graphic formats is displayed
to the left of each component, module, procedure, or source line.

