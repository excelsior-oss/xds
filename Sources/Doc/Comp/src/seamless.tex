\chapter{Seamless Compilation}\label{seamless}
\index{seamless compilation}

We will use the term {\em seamless compilation} if the \xds{}
is configured to call a C compiler, linker or make implicitly
to prepare an executable for your program.
There are several means to achieve seamless compilation:
\begin{enumerate}
\item
        You can define a C compiler invocation command line,
        to call your C compiler after each successful compilation
        of a module, and
        a linker command line, to call linker after successful
        completion of a project (See \ref{seamless:comp}).
\item
        A makefile can be produced after successful completion
        of a project and implicitly called then to compile
        and link your program (See \ref{seamless:make}).
\end{enumerate}

\section{Specifying your C Compiler and Linker}\label{seamless:comp}

The \OERef{COMPILE}
equation determines the command line for
invoking a C compiler. The format of the line is the same as in
template files (See \ref{xc:template}).
An additional argument  MODULE  corresponding to the current module
can be used in the equation\footnote{
  The {\tt EXT} extension specifier cannot be used with the MODULE
  argument.
}.
In the {\bf COMPILE} equation one have to specify:
\begin{itemize}
\item   a name of your C compiler;
\item   C compiler flags, including {\em do compile only} flag;
\item   path to the library header files;
\item   a name of C source file to compile.
\end{itemize}

As a rule, the setting analogous to the following is used
(in the configuration or project file):
\begin{verbatim}
  -COMPILE = "cc -c -I/xds/include %s",module#code;
\end{verbatim}
Then after the successful compilation of the module {\tt A}
\begin{flushleft} \tt
        \XC{} A.mod
\end{flushleft}
the following command line will be invoked:
\begin{flushleft} \tt
cc -c -I/xds/include A.c
\end{flushleft}

{\bf Note:} The \XDS{} uses the file name extension defined
by the {\bf OBJEXT} equation to locate object file.
This equation should be set to the correct value for your system
(default {\bf \Obj}).

The \OERef{LINK}
equation sets the command line which will be invoked after
successful compilation of the project in PROJECT or MAKE operation modes.
As a rule one have to specify in the equation:
\begin{itemize}
\item   a name of your linker;
\item   linker flags;
\item   a set of libraries (and path to the library directory, if required).
\item   a list of modules to link.
\end{itemize}
For example, if your program consists of two modules $A$ and $B$,
where $A$ is the main module and it uses $B$ and project file
({\tt A.prj}) contains following lines:
\begin{verbatim}
-OBJEXT = o
-LINK = "link "; {"%s ",imp+main+oberon#objext;}
\end{verbatim}
then the call
\begin{flushleft} \tt
        \XC{} A =p
\end{flushleft}
will cause the following command to invoke:
\begin{verbatim}
link A.o B.o
\end{verbatim}

In the big project it may be desirable to use separate directories for
different parts of your project. If the option {\bf LONGNAME} is set
ON, the \XDS{} will generate file names according to the current redirection.
E.g. if the following lines are defined in {\tt A.prj}
\begin{verbatim}
-longname+
-lookup = *.o = objects
\end{verbatim}
the command line to invoke the linker will be generated as
\begin{verbatim}
link objects/A.o objects/B.o
\end{verbatim}

\section{Using Make}\label{seamless:make}

Another way to achieve seamless compilation is to generate
makefile (See \ref{xc:modes:gen}) and then call the make
utility using the {\bf LINK} equation (See previous section).  It has
important advantage: a C compiler will be called only when all modules
constituting your program will be compiled.

Following is the sketch of project file:
\begin{verbatim}
-makefile+         % to generate makefile
-mkfname=makefile  % name of generated makefile
-mkfext=           % empty extension for makefile
-template=<name of template file>
-link="make";      % to call make
!module A.mod      % list of modules
\end{verbatim}

Your package contains a sample of a template file
on the BIN directory ({\tt makefile.tem}).

See also options {\bf MODULENAME} and {\bf LONGNAME}.

