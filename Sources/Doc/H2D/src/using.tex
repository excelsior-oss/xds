\chapter{Using H2D}
\label{using}

\section{Headers merging} % !!! Rewrite
\label{using:merge}

A C header file may contain one or more {\tt \#include} directives.
H2D offer the following translation variants for included headers:

\begin{itemize}
\item All headers are merged and translated into a single definition module
      (the \OERef{MERGEALL} option is set ON).
\item Each included header is translated into a separate definition module
      which name is added to the import list (the \OERef{MERGEALL}
      option is set OFF).
\item The headers which have to be merged are explicitly specified using the
      {\tt \#merge} directive (the \OERef{MERGEALL} option must be OFF).
\end{itemize}

If the \OERef{GENSEP} option is set ON, H2D separates pieces of Modula-2
text, which correspond to different merged headers, with comments containing header
file names.

The \PPDir{merge} provides more flexible method of merging control
than the \OERef{MERGEALL} option. The example illustrates situation in which
this directive is very helpful.

\subsection*{Example}

\begin{verbatim}
/* m1.h */                      /* m2.h */
typedef int INTEGER;            struct descriptor{
#include <m2.h>                   INTEGER handl;
#include <m3.h>                 };
#include <m4.h>                 typedef int far * RETVAL;
RETVAL handler();               /* end m2.h */
/* end m1.h */
\end{verbatim}

In this example, the {\tt RETVAL} declaration from {\tt m2.h} is used in
{\tt m1.h}. On the other hand, {\tt m2.h} uses the declaration from {\tt m1.h}
({\tt INTEGER}). Setting the \OERef{MERGEALL} option ON
results in all headers ({\tt m1.h}, {\tt m2.h},
{\tt m3.h}, and {\tt m4.h}) being merged and translated into the single
definition module {\tt m1}. If this is not a desired behaviour,
the \PPDir{merge} should be used instead. If the \OERef{MERGEALL}
option is OFF and the line

\verb'    #merge <m2.h>'

is added to either {\tt m1.h} or the corresponding \verb'#header'
directive in a \ProjectFile{}, H2D produces
{\em three} definition modules {\tt m1}, {\tt m3} and {\tt m4}, where {\tt m1}
is a result of translation of two merged headers {\tt m1.h} and {\tt m2.h}.

\section{Fitting a Modula-2 compiler}
\label{using:fit}

Some of H2D translation rules depend on the target Modula-2 compiler;
even XDS-C and Native XDS require different definition modules.
The \OERef{BACKEND} option is used to reflect the major difference:
whether the target Modula-2 compiler is a native code compiler
(\verb'-BACKEND = Native') or a convertor to C (\verb'-BACKEND = C').
It is also possible to produce definition modules suitable for both XDS-C
and Native XDS (\verb'-BACKEND = Common'). In this case H2D encloses
target-dependent parts with XDS conditional compilation directives.

\subsection{Native code}
\label{using:fit:native}

C headers often contain a number of useful function-like macros.
These macros are translated into procedure declarations with
parameters having type \verb'ARRAY OF SYSTEM.BYTE', which is assignment
compatible with any other type. But C macros exist only at compile-time
and are not present in object files. Therefore, a Modula-2 compiler is
unable to handle them properly unless it is implemented as a convertor to C.
Nevertheless, H2D provides a technique which allows to use C function-like
macros even with a native code Modula-2 compiler.

If the \OERef{BACKEND} option is set to either \verb'Native' or \verb'Common'
and the \OERef{GENMACRO} option is set ON, H2D produces an additional
module containing macro {\it prototypes} --- procedures corresponding to
function-like macros, which bodies consist of a comment with C macro
definition and are expected to be written by a programmer.
These procedures are then declared as external in the main definition module.
Thus, a macro prototype module need not to be imported, it should be just
{\em linked} into an executable which uses the generated definition module.

A macro prototype module name is constructed from a header module name and
a prefix specified by the \OERef{MACPFX} option.

\subsubsection*{Example}

{\ifonline\else\small\fi
\begin{verbatim}
/* macro.h */
    ...
#define cube(x) (x*x*x)
    ...
\end{verbatim}

\Sep %-----------------------------------------------------------------

\begin{verbatim}
(* macro.def  Sep 20  2:38:9  1996 *)
    ...
DEFINITION MODULE ["C"] macro;
    ...
PROCEDURE  / cube ( x: ARRAY OF SYSTEM.BYTE );
    ...
END macro.
\end{verbatim}

\Sep %-----------------------------------------------------------------

\begin{verbatim}
(* m_macro.def  Sep 20  2:38:9  1996 *)
    ...
DEFINITION MODULE m_macro;

IMPORT SYSTEM;
    ...
PROCEDURE ["C"] cube ( x: ARRAY OF SYSTEM.BYTE );
    ...
END m_macro.
\end{verbatim}

\Sep %-----------------------------------------------------------------

\begin{verbatim}
(* m_macro.mod  Sep 20  2:38:9  1996 *)
    ...
IMPLEMENTATION MODULE m_macro;

IMPORT SYSTEM;
    ...
PROCEDURE ["C"] cube ( x: ARRAY OF SYSTEM.BYTE );
(*
#define cube(x) (x*x*x)
*)
BEGIN
END cube;
    ...
END m_macro.
\end{verbatim}
} % end \small

\subsection{Convertor to C}
\label{using:fit:c}

A Modula-2 compiler implemented as a convertor to C (e.g. XDS-C)
converts definition modules written by a programmer to C headers.
But headers corresponding to definition modules generated by H2D
already exist. To prevent them from being overridden, H2D inserts the
{\bf NOHEADER} XDS option, which disables header file generation,
at the beginning of each definition module.

For all included header files, which are not merged (see \ref{using:merge}),
H2D also sets the {\bf CSTDLIB} XDS option according to the parenthesis
used in the \verb'#include' directive -- double quotes or angle brackets.
For top-level header files, this option is set equal to the value of the
\OERef{CSTDLIB} option.

\ifcomment % !!! Do we really need this?

Each definition module has a comment after the import section
similar to the following:

\begin{verbatim}
(* H2D: Required IMPORT clause:
IMPORT a, m1;
*)
\end{verbatim}

In order to use this definition module, you have to add the
commented IMPORT clause to the import section of your module.
If some of the modules listed in it are already imported, you
may wish to remove duplicated names.

\fi

H2D usually has to introduce a number of additional types in the
definition module (see \ref{rules:functions} and \ref{rules:types}).
These types are absent in the original header file, and their usage would
cause C compilation to fail. To solve this problem, H2D constructs
a resulting definition module name from a header file name and a
prefix specified by the \OERef{DEFPFX} option.
Then, it produces a {\em "wrapper" header file}, which name
corresponds to the name of a {\em definition module}, containing
an \verb'#include' directive with original header name, followed by
required type declarations. Type declarations from a \ProjectFile{}
are copied to a wrapper file as well.

\subsubsection*{Example}

{\ifonline\else\small\fi
\begin{verbatim}
/* type.h */

struct Node {
  struct Node *next;
  struct Node *prev;
  int          hash;
};

int Hash(char * str);
\end{verbatim}

\Sep %-----------------------------------------------------------------

\begin{verbatim}
(* h2d_type.def  Sep 20  2:51:7  1996 *)
    ...
DEFINITION MODULE ["C"] h2d_type;

IMPORT SYSTEM;
    ...
<*- GENTYPEDEF *>

TYPE
  PtrNode = POINTER TO Node;

<*+ GENTYPEDEF *>

  Node = RECORD
    next: PtrNode;
    prev: PtrNode;
    hash: SYSTEM.int;
  END;

<*- GENTYPEDEF *>

  PtrSChar = POINTER TO CHAR;

PROCEDURE Hash ( str: PtrSChar ): SYSTEM.int;

END h2d_type.
\end{verbatim}

\Sep %-----------------------------------------------------------------

\begin{verbatim}
/* h2d_type.h  Sep 20  2:51:7  1996 */
    ...
#include "type.h"

#ifndef h2d_type_H_
#define h2d_type_H_

typedef struct Node * PtrNode;
typedef signed char * PtrSChar;

#endif  /* h2d_type_H_ */
\end{verbatim}
} % end \small

\section{Modifying translation rules}
\label{using:modrules}

\subsection{Base types mapping}
\label{using:modrules:mapping}

The \OERef{CTYPE} and \OERef{M2TYPE} options in conjunction with the
\PPDir{variant} provide complete control over mapping of C base types
to \mt{} types.

The \OERef{CTYPE} option specifies sizes (in bytes) of C base types,
and their default mapping to \mt{}:

\verb'    -CTYPE = float              = 4, REAL' \\
\verb'      . . .' \\
\verb'    -CTYPE = unsigned short int = 2, SYSTEM.CARD16'

The \OERef{M2TYPE} option specifies \mt{} types supported by a
particular compiler, with their sizes and families to which they belong:

\verb'    -M2TYPE = CARDINAL        = 4, UNSIGNED' \\
\verb'    -M2TYPE = CHAR            = 1, CHAR' \\
\verb'      . . .' \\
\verb'    -M2TYPE = SYSTEM.SET16    = 2, SET'

Finally, the \PPDir{variant} allows to explicitly specify a \mt{} type for
a particular object:

{\small
\verb'    void f(unsigned short mask)      PROCEDURE f(mask : SYSTEM.SET16);' \\
\verb'    #variant f(0) : SYSTEM.SET16'
} % end \small

{\bf Note:} In order to keep original headers intact, {\tt \#variant}
directives may be placed into the \ProjectFile{} inside the corresponding
\ProjDir{header}.

H2D checks type mappings for correctness using the following rules:
\begin{itemize}
\item type sizes must match
\item floating point C types may only be mapped to \mt{} types defined as REAL
      by \OERef{M2TYPE} option.
\item signed integer C types may be mapped to any \mt{} type except REAL and UNSIGNED.
\item unsigned integer C types may be mapped to any \mt{} type except REAL and SIGNED.
\end{itemize}

One of the most advanced features of this mechanism is the ability
to use \mt{} set types for C objects.
The C programming language, as well as C++, has no built-in set type.
The common practice is to treat unsigned integer types as bit scales
and to use bitwise logical operators to manipulate them. Since \mt{}
provides set types (and no bitwise operators), it would be more convenient to
translate individual integer constants and types to set constants and types.

\subsubsection*{Example}

{\ifonline\else\small\fi
\begin{verbatim}
struct s{
  unsigned int field;
};
typedef unsigned long BITSCALE;
int variable;
void long function(unsigned argument);
char bitarray[10];
#define constant 0x0011

#variant s.field     : BITSET
#variant BITSCALE    : BITSET
#variant variable    : BITSET
#variant function(0) : BITSET
#variant bitarray[]  : SYSTEM.SET8
#variant constant    : SYSTEM.SET16
\end{verbatim}

\Sep %-----------------------------------

\begin{verbatim}
TYPE
  s = RECORD
    field: BITSET;
  END;

  BITSCALE = BITSET;

VAR
  variable: BITSET;

PROCEDURE function ( argument: BITSET );

VAR
  bitarray: ARRAY [0..9] OF SYSTEM.SET8;

CONST
  constant = SYSTEM.SET16{0, 4};
\end{verbatim}
} % end \small

\subsection{Pointer type function parameters}
\label{using:modrules:parameters}

In C, the actual semantics of a pointer type function parameter depends on
that function and cannot be determined automatically. A function
may interpret its parameter of type {\tt T*} as either:

\begin{itemize}
\item pointer to T (type defined as {\tt POINTER TO T})
\item single value passed by reference ({\tt VAR T})
\item array ({\tt ARRAY OF T})
\item array passed by reference ({\tt VAR ARRAY OF T})
\end{itemize}

where the corresponding Modula-2 formal types are given in parenthesis.

The \PPDir{variant} may be used to explicitly point out the semantics
of each pointer type parameter.

{\bf Note:} In order to keep original headers intact, {\tt \#variant}
directives may be placed into the \ProjectFile{} inside the corresponding
\ProjDir{header}.

\subsubsection*{Example}

\begin{verbatim}
#variant function(0) : VAR
#variant function(1) : ARRAY
#variant function(2) : VAR ARRAY

void function(int*, int*, int*, int*);
\end{verbatim}

is translated to:

\begin{verbatim}
TYPE
  PtrSInt = POINTER TO SYSTEM.int;

PROCEDURE function ( VAR arg0: SYSTEM.int;
                         arg1: ARRAY OF SYSTEM.int;
                     VAR arg2: ARRAY OF SYSTEM.int;
                         arg3: PtrSInt );
\end{verbatim}

\subsection{Preserving constant names}
\label{using:modrules:constants}

By default, a \verb'#define' directive introducing a constant is
translated to a constant declaration:

\verb'    #define ENOTEXIST 10              CONST            ' \\
\verb'                                        ENOTEXIST = 10;'

This is the only way in case of a native code \mt{} compiler, since
such constants are substituted by a C preprocessor and do not appear
in object files. But in case of a convertor to C, the original C headers
will be used after conversion and it would be useful to refer to
their names in the generated C text. Setting the \OERef{GENROVARS}
option ON forces constants to be translated to read-only variables
({\bf Note:} this is an XDS language extension). This option has no
effect on generation for a native-code Modula-2 compiler.

The \PPDir{variant} may be used to specify a type for a variable:

\verb'    #define  ENOTEXIST 10             VAR' \\
\verb'    #variant ENOTEXIST : CARDINAL       ENOTEXIST-: CARDINAL;'


