\newif\ifonline\onlinefalse   % set to true automatically by LaTeX OnLine
\newif\ifenglish\englishtrue
%
% def.tex - common settings and useful definitions for XDS manuals
%

%%% Page Setup %%%%%%%%%%%%%%%%%%%%%%%%%%%%%%%%%%%%%%%%%%%%%%%%%%%%%%%%%

\sloppy                             % Do not cross right margin
\raggedbottom                       % No vertical justification

%\setlength{\oddsidemargin}{0 pt}
%\addtolength{\topmargin}{-36 pt}    % Top margin is 0.5"
%\addtolength{\textheight}{36 pt}
%\addtolength{\textwidth}{72 pt}     % Width is 1 inch more than default

%%% Paragraph style %%%%%%%%%%%%%%%%%%%%%%%%%%%%%%%%%%%%%%%%%%%%%%%%%%%%

\setlength{\parindent}{0 pt}
\setlength{\parskip}{6 pt}

%%% Conditional compilation identifiers %%%%%%%%%%%%%%%%%%%%%%%%%%%%%%%%

\newif\ifcomment\commentfalse % never set to true; use for commenting
%\newif\ifonline\onlinefalse   % set to true automatically by LaTeX OnLine

\newif\ifcommon\commonfalse   % Common document for all editions

%-- Used language ------------------------------------------------------

% default is English language
\newif\ifrussianlang    % Russian language

\ifenglish
  \russianlangfalse
\else
  \russianlangtrue
  \ifonline\else
    \def\partname{�����}
    \def\chaptername{�����}
    \def\contentsname{����������}
    \def\listfigurename{���᮪ ��㭪��}
    \def\listtablename{���᮪ ⠡���}
    \def\refname{��뫪�}
    \def\bibname{����������}
    \def\indexname{������}
    \def\figurename{���㭮�}
    \def\tablename{������}
    \def\appendixname{�ਫ������}
  \fi
\fi
 % should select one language

%-- Available front-ends -----------------------------------------------

% default is M2/O2
\newif\iftopspeed       % TopSpeed extensions
\newif\ifjava           % Java
\newif\ifoberon         % Oberon-2

\input{../Common/frontend.tex} % should select one code generator


%-- Available code generators ------------------------------------------

\newif\ifgenc           % C code generation
\newif\ifgencode        % native code generation
\newif\ifgenintel       % Intel x86 code generation
\newif\ifgenvax         % VAX code generation
\newif\ifgendfa         % data flow analysis
\newif\ifgenefd         % Excelsior FlawDetector
\newif\ifgeneit         % Excelsior Installation Toolkit

\input{../Common/backend.tex}  % should select one code generator

\ifgenc\else\ifgendfa\else\ifgenefd\else\ifgeneit\else\gencodetrue\fi\fi\fi\fi

%-- Available platforms ------------------------------------------------

\newif\ifmsdos          % MS-DOS
\newif\ifosii           % OS/2
\newif\ifwinnt          % WinNT & Win95
\newif\iflinux          % Linux
\newif\ifmacosx         % Mac OS X
\newif\ifunix           % Unix

\input{../Common/platform.tex} % should select one platform

\iflinux \unixtrue \fi  % Linux is Unix
\ifmacosx \unixtrue \fi % Mac OS X is Unix

\newcommand{\SkipLine}{
\ifonline
  \newline    % � .hlp ࠡ�⠥� ⮫쪮 /newline, ���쭮� ����������
\else
  \medskip    % � .pdf /newline ���� �訡�� ��� �����
\fi
}

\ifjava
  \ifgencode
    \newcommand{\ProductName}{\mbox{Excelsior JET}}
  \else\ifgenefd
    \newcommand{\ProductName}{\mbox{Excelsior FlawDetector}}
  \else\ifgeneit
    \newcommand{\ProductName}{\mbox{Excelsior Installer}}
  \else
    \JNO
  \fi\fi\fi
\else
  \ifgenc
    \newcommand{\ProductName}{\mbox{XDS-C}}
  \else\ifgencode
    \ifgenintel
      \newcommand{\ProductName}{\mbox{Native XDS-x86}}
    \else\ifgenvax
      \newcommand{\ProductName}{\mbox{Native XDS-VAX}}
    \else
      \newcommand{\ProductName}{\mbox{Native XDS}}
    \fi\fi
  \else
    \JNO
  \fi\fi
\fi


\newif\ifthreads \threadsfalse
\newif\ifdll     \dllfalse
\newif\ifxlink   \xlinkfalse
\newif\ifprofile \profilefalse
\newif\ifxlib    \xlibfalse
\ifgencode
  \ifosii
    \threadstrue\dlltrue\xlinktrue\profiletrue\xlibtrue
  \else\ifwinnt
    \threadstrue\dlltrue\xlinktrue\profiletrue\xlibtrue
  \else\iflinux
  \else\ifmacosx
  \else         \JNO
  \fi\fi\fi\fi
\fi



\newcommand{\BNFsize}{\small}

\newcommand{\XDStm}{XDS}
\newcommand{\XDS}{XDS}
\newcommand{\xds}{XDS}
\newcommand{\Exc}{Excelsior}

%%% File names and extensions %%%%%%%%%%%%%%%%%%%%%%%%%%%%%%%%%%%%%%%%%%

\ifgencode
  \newcommand{\XC}{xc}
  \newcommand{\xc}{{\tt xc}}
\else
  \newcommand{\XC}{xm}
  \newcommand{\xc}{{\tt xm}}
\fi
\newcommand{\XD}{xd}
\newcommand{\xd}{xd}


%!!! warning
% \index of the form \index{xm<..>@{\XC{}<...>}}
% is used in config.tex & usage.tex
% it may lead to wrong order of items in the index.
% I don't know another way to use macro in the index.

%\newcommand{\xc}{{\tt \XC{}}}
\newcommand{\red}{{\tt \xc{}.red}}
\newcommand{\msg}{{\tt \xc{}.msg}}
\newcommand{\cfg}{{\tt \xc{}.cfg}}

\ifgenc
\newcommand{\kwd}{{\tt \xc{}.kwd}}
\fi

\ifgenc
  \newcommand{\Sym}{.sym}
  \newcommand{\Code}{.c}
\fi
\newcommand{\Header}{.h}

\ifgencode
  \newcommand{\Sym}{.sym}
  \newcommand{\Bod}{.bod}
  \ifunix
    \newcommand{\Code}{.o}
  \else
    \newcommand{\Code}{.obj}
  \fi
\fi

\ifunix
  \newcommand{\Obj}{.o}
  \newcommand{\Exe}{}
\else
  \newcommand{\Obj}{.obj}
  \newcommand{\Exe}{.exe}
\fi

\ifunix
  \newcommand{\EXE}{executable}
  \newcommand{\DLL}{shared library}
  \newcommand{\EXEs}{executables}
  \newcommand{\DLLs}{shared libraries}
  \newcommand{\dotDll}{.so}
  \newcommand{\dotExe}{}
\else\ifwinnt
  \newcommand{\EXE}{EXE}
  \newcommand{\DLL}{DLL}
  \newcommand{\EXEs}{EXEs}
  \newcommand{\DLLs}{DLLs}
  \newcommand{\dotObj}{.obj}
  \newcommand{\dotDll}{.dll}
  \newcommand{\dotExe}{.exe}
\else\JNO\fi\fi
\ifunix
  \newcommand{\DirSep}{/}
\else
  \newcommand{\DirSep}{$\backslash$}
\fi

\newcommand{\WinUnix}[2]{\ifwinnt #1\else\ifunix #2\else\JNO\fi\fi{}}

%----------- useful commands -----------%

\newcommand{\EndSection}{\paragraph{}}

\ifrussianlang
  \newcommand{\mt}{\mbox{���㫠-2}}
  \newcommand{\ot}{\mbox{���஭-2}}
  \newcommand{\Seealso}{{\bf ����� ⠪��}\par}
  %\newcommand{\Example}{\paragraph{�ਬ��}}
  %\newcommand{\Examples}{\paragraph{�ਬ���}}
  \newcommand{\Example}{{\bf �ਬ��}\par}
  \newcommand{\Examples}{{\bf �ਬ���}\par}
  \newcommand{\Format}{{\bf ��ଠ�}\par}
  \newcommand{\Description}{{\bf ���ᠭ��}\par}
  \newcommand{\Note}{{\bf ����砭��:}}
  \newcommand{\Warning}{{\bf �।�०�����:}}
\else
  \newcommand{\mt}{\mbox{Modula-2}}
  \newcommand{\ot}{\mbox{Oberon-2}}
  \newcommand{\Seealso}{\begin{flushleft} \bf See also \end{flushleft} }
  \newcommand{\Example}{\paragraph{Example}}
  \newcommand{\Examples}{\paragraph{Examples}}
  \newcommand{\Format}{{\bf Format}\par}
  \newcommand{\Description}{{\bf Description}\par}
\fi

\newcommand{\arrow}{\^{}}
\newcommand{\bs}{$\backslash$}

\ifonline\else
\newcommand{\drawing}[1]{\epsfbox{#1.eps}}
\fi


\newcommand{\maddonly}
{
\ifonline
  \ifrussianlang
    {\footnotesize {\bf ����砭��:} ����㯭� ⮫쪮 �� ����祭��� ��樨 \OERef{M2ADDTYPES}.\par}
  \else
    {\footnotesize NOTE: Only valid when option \OERef{M2ADDTYPES} is set.\par}
  \fi
\else
  \begin{center}
  \begin{tabular}{|l|}
  \hline
  \ifrussianlang
    \footnotesize ���������: ����㯭� ⮫쪮 �� ����祭��� ��樨 \OERef{M2ADDTYPES}.\\
  \else
    \footnotesize NOTE: Only valid when option \OERef{M2ADDTYPES} is set.\\
  \fi
  \hline
  \end{tabular}
  \end{center}
\fi
}

\newcommand{\mextonly}
{
\ifonline
  \ifrussianlang
%    {\footnotesize {\bf ���������:} ����㯭� ⮫쪮 �� ����祭��� ��樨 \OERef{M2EXTENSIONS}.\par}
    {\footnotesize {\bf ����砭��:} ����㯭� ⮫쪮 �� ����祭��� ��樨 \OERef{M2EXTENSIONS}.\par}
  \else
    {\footnotesize NOTE: Only valid when option \OERef{M2EXTENSIONS} is set.\par}
  \fi
\else
  \begin{center}
  \begin{tabular}{|l|}
  \hline
  \ifrussianlang
    \footnotesize ���������: ����㯭� ⮫쪮 �� ����祭��� ��樨 \OERef{M2EXTENSIONS}.\\
  \else
    \footnotesize NOTE: Only valid when option \OERef{M2EXTENSIONS} is set.\\
  \fi
  \hline
  \end{tabular}
  \end{center}
\fi
}

\newcommand{\oextonly}
{
\ifonline
  {\footnotesize NOTE: Only valid when option \OERef{O2EXTENSIONS} is set.\par}
\else
  \begin{center}
  \begin{tabular}{|l|}
  \hline
  \footnotesize NOTE: Only valid when option \OERef{O2EXTENSIONS} is set.\\
  \hline
  \end{tabular}
  \end{center}
\fi
}

\newcommand{\onumextonly}
{
\ifonline
  {\footnotesize NOTE: Only valid when option \OERef{O2NUMEXT} is set.\par}
\else
  \begin{center}
  \begin{tabular}{|l|}
  \hline
  \footnotesize NOTE: Only valid when option \OERef{O2NUMEXT} is set.\\
  \hline
  \end{tabular}
  \end{center}
\fi
}

\newcommand{\lindex}[1]{\protect\index{#1@{\bf #1}}} % !!! Has to be changed

% This command should be used to encapsulate page numbers in the index

\newcommand{\Encap}[1]{{\bf\hyperpage{#1}}}

% Modules

\newcommand{\ModuleI}{}
\newcommand{\ModuleII}{}
\newcommand{\ModuleIII}{}
\newcommand{\ModuleIV}{}
\newif\ifOneModule     % Generate label and index entry prefixed with \ModuleI
\newif\ifTwoModules    % Generate label and index entry prefixed with \ModuleII
\newif\ifFourModules   % Generate label and index entries prefixed with \ModuleIII and \ModuleIV

\newcommand{\NoModules}{
\OneModulefalse
\TwoModulesfalse
\FourModulesfalse
}
\newcommand{\OneModule}{
\OneModuletrue   \label{\ModuleI}   \protect\index{\ModuleI}
\TwoModulesfalse
\FourModulesfalse
}
\newcommand{\TwoModules}{
\OneModuletrue   \label{\ModuleI}   \protect\index{\ModuleI}
\TwoModulestrue  \label{\ModuleII}  \protect\index{\ModuleII}
\FourModulesfalse
}
\newcommand{\FourModules}{
\OneModuletrue   \label{\ModuleI}   \protect\index{\ModuleI}
\TwoModulestrue  \label{\ModuleII}  \protect\index{\ModuleII}
\FourModulestrue \label{\ModuleIII} \protect\index{\ModuleIII}
                 \label{\ModuleIV}  \protect\index{\ModuleIV}
}

\NoModules

% Framed heading

\newif\ifFrameIsDeep       % Frame is subsubsection when online
\newif\ifFrameToContents   % Frames are added to contents

\newcommand{\Frame}[2]{
  \ifonline
    \public{#1}
    \ifFrameIsDeep
      \subsubsection{#1 - #2}
    \else
      \subsection{#1 - #2}
    \fi
  \else
    \bigskip
    \noindent
    \framebox[\textwidth]{\large {\bf #1} \hfill {\sl #2}}
    \ifFrameToContents
      \ifFrameIsDeep
        \addcontentsline{toc}{subsubsection}{#1 - #2}
      \else
        \addcontentsline{toc}{subsection}{#1 - #2}
      \fi
    \fi
  \fi
  \protect\index{#1}
  \ifOneModule
  \label{\ModuleI.#1}
  \protect\index{\ModuleI!#1}
  \fi
  \ifTwoModules
  \label{\ModuleII.#1}
  \protect\index{\ModuleII!#1}
  \fi
  \ifFourModules
  \label{\ModuleIII.#1} \label{\ModuleIV.#1}
  \protect\index{\ModuleIII!#1} \protect\index{\ModuleIV.#1}
  \fi
}

% Module list

\newcommand{\ModuleList}
{\ifonline
\ifTwoModules Modules: \else Module \fi
\IdRef{\ModuleI}\ifTwoModules ,
\IdRef{\ModuleII}\fi\ifFourModules ,
\IdRef{\ModuleIII}, \IdRef{\ModuleIV} \fi
\par
\fi}

% Reference to an id
\newcommand{\IdRef}[1]{{\tt \ifonline \ref[#1]{#1}\else #1\fi}}

% Reference to an unqualified id with qualified label
\newcommand{\IdQualRef}[1]{{\tt \ifonline \ref[#1]{\ModuleI.#1}\else #1\fi}}

% Options/Equations headings

\newcommand{\OptHead}[2]{
  \ifonline
    \subsection{#1 - #2}
    \label{#1}
    \protect\index{#1}
    \protect\index{options!#1}
  \else
    \item[{\bf #1} - #2]
    \protect\index{#1@{\bf #1}|Encap}
    \protect\index{options!#1@{\bf #1}|Encap}
  \fi
}

\newcommand{\EquHead}[2]{
  \ifonline
    \subsection{#1 - #2}
    \label{#1}
    \protect\index{#1}
    \protect\index{equations!#1}
  \else
    \item[#1 - #2]
    \protect\index{#1@{\bf #1}|Encap}
    \protect\index{equations!#1@{\bf #1}|Encap}
  \fi
}

\newcommand{\LinkOpt}[2]{
  \ifonline
    \subsection{/#1 - #2}
    \public{#1}                % ???
    \label{link:options:#1}
    \protect\index{/#1}
    \protect\index{linker options!/#1}
  \else
%    \addcontentsline{toc}{subsection}{\protect\makebox[0.5cm]{}/#1 - #2}
    \protect\index{#1@{\bf /#1}, linker option|Encap}
    \protect\index{linker options!#1@{\bf /#1}|Encap}
    \protect\item[{\large \bf /#1}] \mbox{} \\
  \fi
}

% Reference to an option/equation
\newcommand{\OERef}[1]{\ifonline\ref[#1]{#1}\else{\bf #1}\protect\index{#1@{\bf #1}}\fi}
\newcommand{\OEName}[1]{\ifonline #1\else{\bf #1}\protect\index{#1@{\bf #1}}\fi}

\newcommand{\LORef}[1]{\ifonline\ref[/#1]{link:options:#1}\else
{\bf /#1}\protect\index{#1@{\bf /#1}, linker option}
\fi}

\newcommand{\Ref}[2]{\ifonline\ref[#1]{#2}\else#1\fi}
\ifrussianlang
  \newcommand{\See}[3]{\ifonline\ref[#1]{#3}\else#1 (�. #2\protect\ref{#3})\fi}
\else
  \newcommand{\See}[3]{\ifonline\ref[#1]{#3}\else#1 (see #2\protect\ref{#3})\fi}
\fi

\newcommand{\havetowrite}{{\bf Have to write something here.}}

\newcommand{\EXAMPLE}[2]{
\ifonline
  \par \ref{#2} \par
\else
  \paragraph{Example - #1}
\fi
}

\ifonline\else
  \newcommand{\public}[1]{\label{#1}}
\fi

\newcommand{\Picture}[3]{
\ifonline
  \includegraphics[#2]{#3}
\else
  \includegraphics[#1]{#3.pdf}
\fi
}



\ifonline\title{Oberon-2 Report}\fi

\newif\ifcompact % !!!

\documentstyle[11pt,makeidx,a4]{article}

\makeindex

\begin{document}

%\chapter{The Oberon-2 Report}\label{o2rep}

\begin{center}
{\large\bf The Programming Language Oberon-2} \\
H.M\"ossenb\"ock, N. Wirth  \\
Institut f\"ur Computersysteme, \\
ETH Z\"urich \\
October 1993 \\
\end{center}

\section{Introduction}

Oberon-2 is a general-purpose language in the tradition of Oberon
and Modula-2. Its most important features are block structure, modularity,
separate compilation, static typing with strong type checking (also
across module boundaries), and type extension with type-bound procedures.

Type extension makes Oberon-2 an object-oriented language. An object
is a variable of an abstract data type consisting of private data
(its state) and procedures that operate on this data. Abstract data
types are declared as extensible records. Oberon-2 covers most terms
of object-oriented languages by the established vocabulary of imperative
languages in order to minimize the number of notions for similar concepts.

This report is not intended as a programmer's tutorial. It is intentionally
kept concise. Its function is to serve as a reference for programmers,
implementors, and manual writers. What remains unsaid is mostly left
so intentionally, either because it can be derived from stated rules
of the language, or because it would require to commit the definition
when a general commitment appears as unwise.

Section \ref{o2r:terms}
defines some terms that are used to express the type
checking rules of Oberon-2. Where they appear in the text, they are
written in italics to indicate their special meaning (e.g. the {\em same}
type).

\section{Syntax}

An extended Backus-Naur Formalism (EBNF) is used to describe the syntax
of Oberon-2: Alternatives are separated by \verb+|+. Brackets [ and ] denote
optionality of the enclosed expression, and braces \{ and \} denote
its repetition (possibly 0 times). Non-terminal symbols start with
an upper-case letter (e.g. Statement). Terminal symbols either start
with a lower-case letter (e.g. ident), or are written all in upper-case
letters (e.g. BEGIN), or are denoted by strings (e.g. ":=").

\section{Vocabulary and Representation}

The representation of (terminal) symbols in terms of characters is
defined using the ASCII set. Symbols are identifiers, numbers, strings,
operators, and delimiters. The following lexical rules must be observed:
Blanks and line breaks must not occur within symbols (except in comments,
and blanks in strings). They are ignored unless they are essential
to separate two consecutive symbols. Capital and lower-case letters
are considered as distinct.

\paragraph{\rm 1.} {\em Identifiers} are sequences of letters and
digits. The first character must be a letter.
{\BNFsize
\begin{verbatim}
  ident = letter {letter | digit}.
\end{verbatim}}

\noindent
Examples:
\begin{verbatim}
  x   Scan   Oberon2   GetSymbol   firstLetter
\end{verbatim}

\paragraph{\rm 2.} {\em Numbers} are (unsigned) integer or real constants.
The type of an integer constant is the minimal type to which the constant
value belongs (see \ref{o2r:basic:types}). If the constant
is specified with the suffix H, the representation is hexadecimal
otherwise the representation is decimal.

A real number always contains
a decimal point. Optionally it may also contain a decimal scale factor.
The letter E (or D) means "times ten to the power of". A real number
is of type REAL, unless it has a scale factor containing the letter
D. In this case it is of type LONGREAL.
{\BNFsize
\begin{verbatim}
  number      = integer | real.
  integer     = digit {digit} | digit{hexDigit}"H".
  real        = digit{digit}"."{digit} [ScaleFactor].
  ScaleFactor = ("E" | "D") ["+" | "-"] digit {digit}.
  hexDigit    = digit |"A"|"B"|"C"|"D"|"E"|"F".
  digit       = "0"|"1"|"2"|"3"|"4"|"5"|"6"|"7"|"8"|"9".
\end{verbatim}}

\noindent
Examples:
\begin{verbatim}
  1991           INTEGER          1991
  0DH            SHORTINT         13
  12.3           REAL             12.3
  4.567E8        REAL             456700000
  0.57712566D-6  LONGREAL         0.00000057712566
\end{verbatim}

\paragraph{\rm 3.} {\em Character constants} are denoted by the ordinal
number of the character in hexadecimal notation followed by the letter
X.
{\BNFsize
\begin{verbatim}
  character = digit {hexDigit} "X".
\end{verbatim}}

\paragraph{\rm 4.} {\em Strings} are sequences of characters enclosed
in single (\verb|'|) or double (\verb|"|) quote marks.
The opening quote must be
the same as the closing quote and must not occur within the string.
The number of characters in a string is called its length. A string
of length 1 can be used wherever a character constant is allowed and
vice versa.
{\BNFsize
\begin{verbatim}
  string = '"' {char} '"' | "'" {char} "'".
\end{verbatim}}

\noindent
Examples:
\begin{verbatim}
  "Oberon-2"     "Don't worry!"   "x"
\end{verbatim}

\paragraph{\rm 5.} {\em Operators} and {\em delimiters} are the special
characters, character pairs, or reserved words listed below. The reserved
words consist exclusively of capital letters and cannot be used as
identifiers.
\begin{verbatim}
  +         :=         ARRAY    IMPORT     RETURN
  -         ^          BEGIN    IN         THEN
  *         =          BY       IS         TO
  /         #          CASE     LOOP       TYPE
  ~         <          CONST    MOD        UNTIL
  &         >          DIV      MODULE     VAR
  .         <=         DO       NIL        WHILE
  ,         >=         ELSE     OF         WITH
  ;         ..         ELSIF    OR
  |         :          END      POINTER
  (         )          EXIT     PROCEDURE
  [         ]          FOR      RECORD
  {         }          IF       REPEAT
\end{verbatim}

\paragraph{\rm 6.} {\em Comments} may be inserted between any two
symbols in a program. They are arbitrary character sequences opened
by the bracket \verb|(*| and closed by \verb|*)|.
Comments may be nested. They do not affect the meaning of a program.

\section{Declarations and scope rules}\label{o2r:decls}

Every identifier occurring in a program must be introduced by a declaration,
unless it is a predeclared identifier. Declarations also specify certain
permanent properties of an object, such as whether it is a constant,
a type, a variable, or a procedure. The identifier is then used to
refer to the associated object.

The {\em  scope} of an object $x$ extends textually from the point of
its declaration to the end of the {\em block} (module, procedure,
or record) to which the declaration belongs and hence to which the
object is {\em local.} It excludes the scopes of equally named objects
which are declared in nested blocks. The scope rules are:

\begin{enumerate}
\item
    No identifier may denote more than one object within
    a given scope (i.e. no identifier may be declared twice in a block);
\item
    An object may only be referenced within its scope;
\item
    A type $T$ of the form POINTER TO $T_1$ (see \ref{o2r:pointer})
    can be declared before the scope of $T_1$.
    The declaration of $T_1$ must follow in the same block to which
    $T$ is local;
\item
    Identifiers denoting record fields (see \ref{o2r:record})
    or type-bound procedures (see \ref{o2r:type:bound})
    are valid in record designators only.
\end{enumerate}

\noindent
An identifier declared in a module block may be followed by an export
mark (\verb|"*"| or \verb|"-"|) in its declaration to indicate that
it is exported.
An identifier $x$ exported by a module $M$
may be used in other modules,
if they import $M$ (see section \ref{o2r:module}). The identifier
is then denoted as $M.x$ in these modules and is called a {\em qualified
identifier}. Identifiers marked with \verb|"-"| in their declaration
are {\em read-only} in importing modules.
{\BNFsize
\begin{verbatim}
  Qualident = [ident "."]ident.
  IdentDef  = ident [" * " | " - "].
\end{verbatim}}

The following identifiers are predeclared; their meaning is defined
in the indicated sections:
\begin{center}
\ifonline
\begin{tabular}{llll}
\else
\begin{tabular}{ll@{\hspace{2.0cm}}ll}
\fi
ABS     & (\ref{o2r:predecl:proc}) &  LEN      & (\ref{o2r:predecl:proc}) \\
ASH     & (\ref{o2r:predecl:proc}) &  LONG     & (\ref{o2r:predecl:proc}) \\
BOOLEAN & (\ref{o2r:basic:types})  &  LONGINT  & (\ref{o2r:basic:types})  \\
CAP     & (\ref{o2r:predecl:proc}) &  LONGREAL & (\ref{o2r:basic:types})  \\
CHAR    & (\ref{o2r:basic:types})  &  MAX      & (\ref{o2r:predecl:proc}) \\
CHR     & (\ref{o2r:predecl:proc}) &  MIN      & (\ref{o2r:predecl:proc}) \\
COPY    & (\ref{o2r:predecl:proc}) &  NEW      & (\ref{o2r:predecl:proc}) \\
DEC     & (\ref{o2r:predecl:proc}) &  ODD      & (\ref{o2r:predecl:proc}) \\
ENTIER  & (\ref{o2r:predecl:proc}) &  ORD      & (\ref{o2r:predecl:proc}) \\
EXCL    & (\ref{o2r:predecl:proc}) &  REAL     & (\ref{o2r:basic:types})  \\
FALSE   & (\ref{o2r:basic:types})  &  SET      & (\ref{o2r:basic:types})  \\
HALT    & (\ref{o2r:predecl:proc}) &  SHORT    & (\ref{o2r:predecl:proc}) \\
INC     & (\ref{o2r:predecl:proc}) &  SHORTINT & (\ref{o2r:basic:types})  \\
INCL    & (\ref{o2r:predecl:proc}) &  SIZE     & (\ref{o2r:predecl:proc}) \\
INTEGER & (\ref{o2r:basic:types})  &  TRUE     & (\ref{o2r:basic:types})  \\
\end{tabular}
\end{center}

\section{Constant declarations}

A constant declaration associates an identifier with a constant value.
{\BNFsize
\begin{verbatim}
  ConstantDeclaration = IdentDef "=" ConstExpression.
  ConstExpression     = Expression.
\end{verbatim}}
A constant expression is an expression that can be evaluated by a
mere textual scan without actually executing the program. Its operands
are constants (section \ref{o2r:expr}) or predeclared
functions (section \ref{o2r:predecl:proc}) that can be evaluated
at compile time.
Examples of constant declarations are:
\begin{verbatim}
  N = 100
  limit = 2*N - 1
  fullSet = {MIN(SET)..MAX(SET)}
\end{verbatim}

\section{Type declarations}\label{o2r:type:dcl}

A data type determines the set of values which variables of that type
may assume, and the operators that are applicable. A type declaration
associates an identifier with a type. In the case of structured types
(arrays and records) it also defines the structure of variables of
this type.
{\BNFsize
\begin{verbatim}
  TypeDeclaration = IdentDef "=" Type.
  Type            = Qualident | ArrayType | RecordType |
                    PointerType | ProcedureType.
\end{verbatim}}

\noindent
Examples:
\begin{verbatim}
  Table = ARRAY N OF REAL
  Tree =  POINTER TO Node
  Node =  RECORD
    key: INTEGER;
    left, right: Tree
  END
  CenterTree = POINTER TO CenterNode
  CenterNode = RECORD (Node)
    width: INTEGER;
    subnode: Tree
  END
  Function = PROCEDURE(x: INTEGER): INTEGER
\end{verbatim}

\subsection{Basic types}\label{o2r:basic:types}

The basic types are denoted by predeclared identifiers. The associated
operators are defined in \ref{o2r:operators} and the predeclared
function procedures in \ref{o2r:predecl:proc} The values
of the given basic types are the following:
\begin{flushleft}
\ifonline
\begin{tabular}{lll}
\else
\begin{tabular}{llp{7cm}}
\fi
1. & { BOOLEAN}  &   the truth values {\tt TRUE} and {\tt FALSE}               \\
2. & { CHAR}     &   the characters of the extended ASCII set
                                                     {\tt (0X..0FFX)}    \\
3. & { SHORTINT} &   the integers between {\tt MIN(SHORTINT)}
                                      and {\tt MAX(SHORTINT)}     \\
4. & { INTEGER}  &   the integers between {\tt MIN(INTEGER)}
                                                and {\tt MAX(INTEGER)}      \\
5. & { LONGINT}  &   the integers between {\tt MIN(LONGINT)}
                                                and {\tt MAX(LONGINT)}      \\
6. & { REAL}     &   the real numbers between {\tt MIN(REAL)}
                                                and {\tt MAX(REAL)}         \\
7. & { LONGREAL} &   the real numbers between {\tt MIN(LONGREAL)}
                                                and {\tt MAX(LONGREAL)}     \\
8. & { SET}      &   the sets of integers between 0 and {\tt MAX(SET)}   \\
\end{tabular}
\end{flushleft}
Types 3 to 5 are {\em integer types}, types 6 and 7 are {\em real
types}, and together they are called {\em numeric types}. They
form a hierarchy; the larger type {\em includes} (the values of)
the smaller type:
$$
\mbox{LONGREAL} \supseteq
\mbox{REAL} \supseteq
\mbox{LONGINT} \supseteq
\mbox{INTEGER} \supseteq
\mbox{SHORTINT}
$$

\subsection{Array types}

An array is a structure consisting of a number of elements which are
all of the same type, called the {\em element type}. The number of
elements of an array is called its {\em length}. The elements of
the array are designated by indices, which are integers between 0
and the length minus 1.
{\BNFsize
\begin{verbatim}
  ArrayType = ARRAY [ Length { "," Length}] OF Type.
  Length    = ConstExpression.
\end{verbatim}}

A type of the form
\begin{verbatim}
   ARRAY L0, L1, ..., Ln OF T
\end{verbatim}
is understood as an abbreviation of
\begin{verbatim}
   ARRAY L0 OF
     ARRAY L1 OF
     ...
       ARRAY Ln OF T
\end{verbatim}
Arrays declared without length are called {\em open arrays}. They
are restricted to pointer base types (see \ref{o2r:pointer}),
element types of open array types, and formal parameter types
(see \ref{o2r:formal:param}).
Examples:
\begin{verbatim}
   ARRAY 10, N OF INTEGER
   ARRAY OF CHAR
\end{verbatim}

\subsection{Record types}\label{o2r:record}

A record type is a structure consisting of a fixed number of elements,
called {\em fields}, with possibly different types. The record type
declaration specifies the name and type of each field. The scope of
the field identifiers extends from the point of their declaration
to the end of the record type, but they are also visible within designators
referring to elements of record variables (see \ref{o2r:operands}).
If a record type is exported, field identifiers that are to be visible
outside the declaring module must be marked. They are called {\em public
fields}; unmarked elements are called {\em private fields}.
{\BNFsize
\begin{verbatim}
RecordType = RECORD ["("BaseType")"]
               FieldList{";" FieldList}
             END.
BaseType   = Qualident.
FieldList  = [IdentList ":" Type ].
\end{verbatim}}
Record types are extensible, i.e. a record type can be declared as
an extension of another record type. In the example
\begin{verbatim}
   T0 = RECORD x: INTEGER END
   T1 = RECORD (T0) y: REAL END
\end{verbatim}
$T_1$ is a (direct) {\em extension} of $T_0$ and $T_0$ is the (direct)
{\em base type} of $T_1$ (see section \ref{o2r:terms}). An extended
type $T_1$ consists of the fields of its base type and of the fields
which are declared in $T_1$ (see section \ref{o2r:type:dcl}).
All identifiers declared in the extended record must be different
from the identifiers declared in its base type record(s).

\noindent
Examples of record type declarations:
\begin{verbatim}
  RECORD
    day, month, year: INTEGER
  END

  RECORD
    name, firstname: ARRAY 32 OF CHAR;
    age: INTEGER;
    salary: REAL
  END
\end{verbatim}

\subsection{Pointer types}\label{o2r:pointer}

Variables of a pointer type $P$ assume as values pointers to variables
of some type $T$. $T$ is called the pointer base type of $P$ and
must be a record or array type. Pointer types adopt the extension
relation of their pointer base types: if a type $T_1$ is an extension
of $T$, and $P_1$ is of type {\tt POINTER TO $T_1$},
then $P_1$ is also an extension of $P$.
{\BNFsize
\begin{verbatim}
  PointerType = POINTER TO Type.
\end{verbatim}}
If $p$ is a variable of type {\tt $P$ = POINTER TO $T$}, a call of the
predeclared procedure {\tt NEW(p)} (see \ref{o2r:predecl:proc})
allocates a variable of type $T$ in free storage. If $T$ is a record type
or an array type with fixed length, the allocation has to be done
with {\tt NEW(p)}; if $T$ is an $n$-dimensional open array type the allocation
has to be done with
{\tt NEW($p,e_0,\ldots,e_{n-1}$)} where $T$ is allocated with
lengths given by the expressions $e_0,\ldots,e_{n-1}$.
\index{memory management}
In either case a pointer
to the allocated variable is assigned to $p$. $p$ is of type $P$.
The {\em referenced}
variable $p$\arrow (pronounced as {\em $p$-referenced}) is of type $T$.
Any pointer
variable may assume the value NIL, which points to no variable at
all.

\subsection{Procedure types}\label{o2r:proctype}

Variables of a procedure type $T$ have a procedure (or NIL) as value.
If a procedure $P$ is assigned to a variable of type $T$, the formal parameter
lists (see section \ref{o2r:formal:param}) of $P$ and $T$ must
{\em match} (see \ref{o2r:terms}). $P$ must not be a
predeclared or type-bound procedure nor may it be local to another
procedure.
{\BNFsize
\begin{verbatim}
  ProcedureType  = PROCEDURE [FormalParameters].
\end{verbatim}}

\section{Variable declarations}\label{o2r:var}

Variable declarations introduce variables by defining an identifier
and a data type for them.
{\BNFsize
\begin{verbatim}
  VariableDeclaration = IdentList ":" Type.
\end{verbatim}}
Record and pointer variables have both a {\em static} type (the type
with which they are declared - simply called their type) and a {\em dynamic}
type (the type they assume at run time). For pointers and variable
parameters of record type the dynamic type may be an extension of
their static type. The static type determines which fields of a record
are accessible. The dynamic type is used to call type-bound procedures
(see \ref{o2r:type:bound}).

\noindent
Examples of variable declarations (refer to examples in \ref{o2r:type:dcl}):
\begin{verbatim}
  i, j, k: INTEGER
  x, y: REAL
  p, q: BOOLEAN
  s: SET
  F: Function
  a: ARRAY 100 OF REAL
  w: ARRAY 16 OF RECORD
       name  : ARRAY 32 OF CHAR;
       ccount: INTEGER
     END
  t, c: Tree
\end{verbatim}

\section{Expressions}\label{o2r:expr}

Expressions are constructs denoting rules of computation
whereby constants
and current values of variables are combined to compute other values
by the application of operators and function procedures. Expressions
consist of operands and operators. Parentheses may be used to express
specific associations of operators and operands.

\subsection{Operands}\label{o2r:operands}

With the exception of set constructors and literal constants
(numbers,
character constants, or strings), operands are denoted by {\em designators}.
A designator consists of an identifier referring to a constant, variable,
or procedure. This identifier may possibly be qualified by a module
identifier (see sections \ref{o2r:decls} and \ref{o2r:module})
and may be followed by selectors if the designated
object is an element of a structure.
{\BNFsize
\begin{verbatim}
  Designator     = Qualident { "." ident |
                   "[" ExpressionList "]" |
                   "^" | "(" Qualident ")" }.
  ExpressionList = Expression {"," Expression}.
\end{verbatim}}
If a designates an array, then $a[e]$ denotes that element of a whose
index is the current value of the expression $e$. The type of $e$ must
be an integer type. A designator of the form $a[e_0,e_1,\ldots,e_n]$ stands
for $a[e_0][e_1]\ldots[e_n]$.
If $r$ designates a record, then $r.f$ denotes the field $f$ of $r$ or the
procedure $f$ bound to the dynamic type of $r$
(section \ref{o2r:type:bound}).
If $p$ designates a pointer, $p$\arrow denotes
the variable which is referenced
by p. The designators $p$\arrow$.f$ and $p$\arrow$[e]$
may be abbreviated as $p.f$ and
$p[e]$, i.e. record and array selectors imply dereferencing.
If $a$ or $r$ are read-only, then also $a[e]$ and $r.f$ are read-only.

A {\em type guard } $v(T)$ asserts that the dynamic type of $v$ is $T$ (or
an extension of $T$), i.e. program execution is aborted, if the dynamic
type of $v$ is not $T$ (or an extension of $T$). Within the designator,
$v$ is then regarded as having the static type $T$. The guard is applicable,
if
\begin{enumerate}
\item
$v$ is a variable parameter of record type or $v$ is a pointer, and
if
\item
$T$ is an extension of the static type of $v$
\end{enumerate}

If the designated object is a constant or a variable, then the designator
refers to its current value. If it is a procedure, the designator
refers to that procedure unless it is followed by a (possibly empty)
parameter list in which case it implies an activation of that procedure
and stands for the value resulting from its execution. The actual
parameters must correspond to the formal parameters as in proper procedure
calls (see \ref{o2r:formal:param}).

\noindent
Examples of designators (refer to examples in \ref{o2r:var}):
\begin{verbatim}
  i                       (INTEGER)
  a[i]                    (REAL)
  w[3].name[i]            (CHAR)
  t.left.right            (Tree)
  t(CenterNode).subnode   (Tree)
\end{verbatim}

\subsection{Operators}\label{o2r:operators}

Four classes of operators with different precedences (binding strengths)
are syntactically distinguished in expressions. The operator \verb|~| has
the highest precedence, followed by multiplication operators, addition
operators, and relations. Operators of the same precedence associate
from left to right. For example, \verb|x-y-z| stands for \verb|(x-y)-z|.
{\BNFsize
\begin{verbatim}
  Expression       = SimpleExpression
                     [ Relation SimpleExpression].
  SimpleExpression = ["+" | "-"] Term {AddOperator Term}.
  Term             = Factor {MulOperator Factor}.
  Factor           = Designator [ActualParameters] |
                     number | character | string | NIL |
                     Set | "(" Expression ")" | "~" Factor.
  Set              = "{"[Element {","Element}]"}".
  Element          = Expression [".."Expression].
  ActualParameters = "(" [ExpressionList] ")".
  Relation         = "=" | "#" | "<" | "<=" | ">" | ">=" |
                     IN | IS.
  AddOperator      = "+" | "-" | OR.
  MulOperator      = "*" | "/" | DIV | MOD | "&".
\end{verbatim}}

\noindent
The available operators are listed in the following tables. Some operators
are applicable to operands of various types, denoting different operations.
In these cases, the actual operation is identified by the type of
the operands. The operands must be {\em expression compatible} with
respect to the operator (see \ref{o2r:terms}).

\ifonline
\subsection{Logical operators}
\else
\subsubsection{\em Logical operators}
\fi

\begin{tabular}{llll}
\verb|OR|  & logical disjunction & \verb|p OR q| & "if p then TRUE, else q" \\
\verb|&|   & logical conjunction & \verb|p & q|  &  "if p then q, else FALSE" \\
\verb|~|   & negation            & \verb|~p|     &         "not p"  \\
\end{tabular}

\noindent
These operators apply to BOOLEAN operands and yield a BOOLEAN result.

\ifonline
\subsection{Arithmetic operators}
\else
\subsubsection{\em Arithmetic operators}
\fi

\begin{tabular}{ll}
        \verb|+|   & sum                 \\
        \verb|-|   & difference          \\
        \verb|*|   & product             \\
        \verb|/|   & real quotient       \\
        \verb|DIV| & integer quotient    \\
        \verb|MOD| & modulus             \\
\end{tabular}

\noindent
The operators \verb|+|, \verb|-|, \verb|*|, and \verb|/|
apply to operands of numeric types. The
type of the result is the type of that operand which includes the
type of the other operand, except for division (\verb|/|), where the result
is the smallest real type which includes both operand types. When
used as monadic operators, - denotes sign inversion and + denotes
the identity operation. The operators DIV and MOD apply to integer
operands only. They are related by the following formulas defined
for any x and positive divisors y:
\begin{verbatim}
  x = (x DIV y) * y + (x MOD y)
  0 <= (x MOD y) < y
\end{verbatim}

\noindent
Examples:
\begin{flushleft}
\begin{tabular}{cccc}
x  &  y  & x DIV y & x MOD y \\
5  &  3  &   1     &    2    \\
-5 &  3  &  -2     &    1    \\
\end{tabular}
\end{flushleft}

\ifonline
\subsection{Set Operators}
\else
\subsubsection{\em Set Operators}
\fi

\begin{tabular}{ll}
        \verb|+| &  union                                            \\
        \verb|-| &  difference  (x - y = x * (-y))                   \\
        \verb|*| &  intersection                                     \\
        \verb|/| &  symmetric set difference (x / y = (x-y) + (y-x)) \\
\end{tabular}

\noindent
Set operators apply to operands of type SET and yield a result of
type SET. The monadic minus sign denotes the complement of x, i.e.
-x denotes the set of integers between 0 and MAX(SET) which are not
elements of x. Set operators are not associative (\verb|(a+b)-c # a+(b-c)|).

A set constructor defines the value of a set by listing its elements
between curly brackets. The elements must be integers in the range
0..MAX(SET). A range a..b denotes all integers in the interval [a,
b].

\ifonline
\subsection{Relations}
\else
\subsubsection{\em Relations}
\fi

\begin{tabular}{ll}
        \verb|=| &  equal               \\
        \verb|#| &  unequal             \\
        \verb|<| &  less                \\
        \verb|<=|&  less or equal       \\
        \verb|>| &  greater             \\
        \verb|>=|&  greater or equal    \\
        \verb|IN|&  set membership      \\
        \verb|IS|&  type test           \\
\end{tabular}

\noindent
Relations yield a BOOLEAN result. The relations \verb|=|,
\verb|#|, \verb|<|, \verb|<=|, \verb|>| and \verb|>=|
apply to the numeric types, CHAR, strings, and character arrays containing
0X as a terminator.
The relations \verb|=| and \verb|#|
also apply to BOOLEAN and SET, as well
as to pointer and procedure types (including the value NIL).
\verb|x IN s| stands for "x is an element of s". x must be of an integer
type, and s of type SET.
\verb|v IS T| stands for "the dynamic type of v is T (or an extension of
T)" and is called a {\em type test}. It is applicable if
\begin{enumerate}
\item
     v is a variable parameter of record type or v is
     a pointer, and if
\item
     T is an extension of the static type of v
\end{enumerate}

\noindent
Examples of expressions (refer to examples in \ref{o2r:var}):
\begin{verbatim}
  1991                    INTEGER
  i DIV 3                 INTEGER
  ~p OR q                 BOOLEAN
  (i+j) * (i-j)           INTEGER
  s - {8, 9, 13}          SET
  i + x                   REAL
  a[i+j] * a[i-j]         REAL
  (0<=i) & (i<100)        BOOLEAN
  t.key = 0               BOOLEAN
  k IN {i..j-1}           BOOLEAN
  w[i].name <= "John"     BOOLEAN
  t IS CenterNode         BOOLEAN
\end{verbatim}

\section{Statements}

Statements denote actions. There are elementary and structured statements.
Elementary statements are not composed of any parts that are themselves
statements. They are the {\bf assignment}, the {\bf procedure call},
the {\bf return}, and the {\bf exit} statement. Structured statements
are composed of parts that are themselves statements. They are used
to express sequencing and conditional, selective, and repetitive execution.
A statement may also be empty, in which case it denotes no action.
The empty statement is included in order to relax punctuation rules
in statement sequences.
{\BNFsize
\begin{verbatim}
  Statement =
    [ Assignment | ProcedureCall | IfStatement |
      CaseStatement | WhileStatement | RepeatStatement |
      ForStatement | LoopStatement | WithStatement |
      EXIT | RETURN [Expression]
    ].
\end{verbatim}}

\subsection{Assignments}

Assignments replace the current value of a variable by a new value
specified by an expression. The expression must be {\em assignment
compatible} with the variable (see \ref{o2r:terms}).
The assignment operator is written as ":=" and pronounced as becomes.
{\BNFsize
\begin{verbatim}
  Assignment = Designator ":=" Expression.
\end{verbatim}}

If an expression $e$ of type $T_e$ is assigned to a variable $v$ of type
$T_v$, the following happens:
\begin{enumerate}
\item
if $T_v$ and $T_e$ are record types, only those fields
of $T_e$ are assigned which also belong to $T_v$ ({\em projection}); the
dynamic type of $v$ must be the {\em same} as the static type of $v$ and is
not changed by the assignment;
\item
if $T_v$ and $T_e$ are pointer types, the dynamic type
of $v$ becomes the dynamic type of $e$;
\item
if $T_v$ is ARRAY n OF CHAR and $e$ is a string of length
$m < n$, $v[i]$ becomes $e_i$ for $i=0..m-1$ and $v[m]$ becomes $0X$.
\end{enumerate}

\noindent
Examples of assignments (refer to examples in \ref{o2r:var}):
\begin{flushleft}
\begin{tabular}{ll}
\tt i :=  0                             \\
\tt p := i = j                          \\
\tt x := i + 1                          \\
\tt k := log2(i+j)                      \\
\tt F := log2                  & (* see \ref{o2r:formal:param} *) \\
\verb|s := {2, 3, 5, 7, 11, 13}|           \\
\tt a[i] := (x+y) * (x-y)               \\
\tt t.key := i                          \\
\tt w[i+1].name := "John"               \\
\tt t := c                              \\
\end{tabular}
\end{flushleft}

\subsection{Procedure calls}

A procedure call activates a procedure. It may contain a list of actual
parameters which replace the corresponding formal parameters defined
in the procedure declaration (see section \ref{o2r:proctype}).
The correspondence is established by the positions of the parameters
in the actual and formal parameter lists. There are two kinds of parameters:
{\em variable} and {\em value parameters}.

        If a formal parameter is a variable parameter, the corresponding
actual parameter must be a designator denoting a variable. If it denotes
an element of a structured variable, the component selectors are evaluated
when the formal/actual parameter substitution takes place, i.e. before
the execution of the procedure. If a formal parameter is a value parameter,
the corresponding actual parameter must be an expression. This expression
is evaluated before the procedure activation, and the resulting value
is assigned to the formal parameter (see also \ref{o2r:formal:param}).
{\BNFsize
\begin{verbatim}
  ProcedureCall = Designator [ActualParameters].
\end{verbatim}}

\noindent
Examples:
\begin{flushleft}
\begin{tabular}{ll}
\verb|WriteInt(i*2+1)|          & (* see \ref{o2r:formal:param} *) \\
\verb|INC(w[k].count)|                                         \\
\verb|t.Insert("John")|         & (* see \ref{o2r:module} *)   \\
\end{tabular}
\end{flushleft}

\subsection{Statement sequences}

Statement sequences denote the sequence of actions specified by the
component statements which are separated by semicolons.
{\BNFsize
\begin{verbatim}
  StatementSequence = Statement {";" Statement}.
\end{verbatim}}

\subsection{If statements}
{\BNFsize
\begin{verbatim}
  IfStatement =
         IF Expression THEN StatementSequence
       { ELSIF Expression THEN StatementSequence }
       [ ELSE StatementSequence ]
         END.
\end{verbatim}}
If statements specify the conditional execution of guarded
statement sequences. The Boolean expression preceding a statement
sequence is called its {\em guard.} The guards are evaluated in sequence
of occurrence, until one evaluates to TRUE, whereafter its associated
statement sequence is executed. If no guard is satisfied, the statement
sequence following the symbol ELSE is executed, if there is one.

\noindent
Example:
\begin{verbatim}
  IF (ch >= "A") & (ch <= "Z") THEN ReadIdentifier
  ELSIF (ch >= "0") & (ch <= "9") THEN ReadNumber
  ELSIF (ch = " ' ") OR (ch = '"') THEN ReadString
  ELSE SpecialCharacter
  END;
\end{verbatim}

\subsection{Case statements}

Case statements specify the selection and execution of a statement
sequence according to the value of an expression. First the case expression
is evaluated, then that statement sequence is executed whose case
label list contains the obtained value. The case expression must either
be of an {\em integer} type that {\em includes} the types of all case labels,
or both the case expression and the case labels must be of type CHAR.
Case labels are constants, and no value must occur more than once.
If the value of the expression does not occur as a label of any case,
the statement sequence following the symbol ELSE is selected, if there
is one, otherwise the program is aborted.
{\BNFsize
\begin{verbatim}
  CaseStatement = CASE Expression OF Case {"|" Case}
                  [ ELSE StatementSequence ]
                  END.
  Case          = [CaseLabelList ":" StatementSequence].
  CaseLabelList = CaseLabels {"," CaseLabels}.
  CaseLabels    = ConstExpression [ ".." ConstExpression].
\end{verbatim}}

\noindent
Example:
\begin{verbatim}
  CASE ch OF
    "A" .. "Z": ReadIdentifier
  | "0" .. "9": ReadNumber
  | "'", '"'  : ReadString
  ELSE SpecialCharacter
  END
\end{verbatim}

\subsection{While statements}

While statements specify the repeated execution of a statement sequence
while the Boolean expression (its {\em guard}) yields TRUE. The guard
is checked before every execution of the statement sequence.
{\BNFsize
\begin{verbatim}
  WhileStatement = WHILE Expression DO StatementSequence END.
\end{verbatim}}

\noindent
Examples:
\begin{verbatim}
  WHILE i > 0 DO i := i DIV 2; k := k + 1 END
  WHILE (t # NIL) & (t.key # i) DO t := t.left END
\end{verbatim}

\subsection{Repeat statements}

A repeat statement specifies the repeated execution of a statement
sequence until a condition specified by a Boolean expression is satisfied.
The statement sequence is executed at least once.
{\BNFsize
\begin{verbatim}
  RepeatStatement = REPEAT StatementSequence UNTIL Expression.
\end{verbatim}}

\subsection{For statements}

A for statement specifies the repeated execution of a statement sequence
for a fixed number of times while a progression of values is assigned
to an integer variable called the {\em control variable} of the for
statement.
{\BNFsize
\begin{verbatim}
  ForStatement = FOR ident":="Expression TO Expression
                 [ BY ConstExpression ] DO StatementSequence
                 END.
\end{verbatim}}
The statement
\begin{verbatim}
  FOR v := beg TO end BY step DO statements END
\end{verbatim}
is equivalent to
\begin{verbatim}
  temp := end; v := beg;
  IF step > 0 THEN
    WHILE v <= temp DO statements; v := v + step END
  ELSE
    WHILE v >= temp DO statements; v := v + step END
  END;
\end{verbatim}
{\tt temp} has the {\em same} type as {\tt v}.
{\tt step} must be a non-zero constant expression.
If $step$ is not specified, it is assumed to be 1.

\noindent
Examples:
\begin{verbatim}
  FOR i := 0 TO 79 DO k := k + a[i] END
  FOR i := 79 TO 1 BY -1 DO a[i] := a[i-1] END
\end{verbatim}

\subsection{Loop statements}

A loop statement specifies the repeated execution of a statement sequence.
It is terminated upon execution of an exit statement within that sequence
(see \ref{o2r:return}).
{\BNFsize
\begin{verbatim}
  LoopStatement = LOOP StatementSequence END.
\end{verbatim}}

\noindent
Example:
\begin{verbatim}
  LOOP
    ReadInt(i);
    IF i < 0 THEN EXIT END;
    WriteInt(i)
  END
\end{verbatim}
Loop statements are useful to express repetitions with several exit
points or cases where the exit condition is in the middle of the repeated
statement sequence.

\subsection{Return and exit statements}\label{o2r:return}

A return statement indicates the termination of a procedure. It is
denoted by the symbol RETURN, followed by an expression if the procedure
is a function procedure. The type of the expression must be {\em assignment
compatible} (see \ref{o2r:terms}) with the result type
specified in the procedure heading (see section \ref{o2r:proc:dcl}).

Function procedures require the presence of a return statement indicating
the result value. In proper procedures, a return statement is implied
by the end of the procedure body. Any explicit return statement therefore
appears as an additional (probably exceptional) termination point.

An exit statement is denoted by the symbol EXIT. It specifies termination
of the enclosing loop statement and continuation with the statement
following that loop statement. Exit statements are contextually, although
not syntactically associated with the loop statement which contains
them.

\subsection{With statements}

With statements execute a statement sequence depending on the result
of a type test and apply a type guard to every occurrence of the tested
variable within this statement sequence.
{\BNFsize
\begin{verbatim}
  WithStatement = WITH Guard DO StatementSequence
                  { "|" Guard DO StatementSequence }
                  [ ELSE StatementSequence ]
                  END.
  Guard         = Qualident ":" Qualident.
\end{verbatim}}
If $v$ is a variable parameter of record type or a pointer variable,
and if it is of a static type $T_0$, the statement
\begin{verbatim}
  WITH v: T1 DO S1 |v: T2 DO S2 ELSE S3 END
\end{verbatim}
has the following meaning: if the dynamic type of $v$ is $T_1$, then the
statement sequence $S_1$ is executed where $v$ is regarded as if it had
the static type $T_1$; else if the dynamic type of $v$ is $T_2$, then $S_2$ is
executed where $v$ is regarded as if it had the static type $T_2$; else
$S_3$ is executed. $T_1$ and T2 must be extensions of $T_0$. If no type test
is satisfied and if an else clause is missing the program is aborted.

\noindent
Example:
\begin{verbatim}
  WITH t: CenterTree DO i := t.width; c := t.subnode END
\end{verbatim}

\section{Procedure declarations}\label{o2r:proc:dcl}

A procedure declaration consists of a procedure heading and a procedure
body. The heading specifies the procedure identifier and the formal
parameters. For type-bound procedures it also specifies the receiver
parameter. The body contains declarations and statements. The procedure
identifier is repeated at the end of the procedure declaration.

There are two kinds of procedures: {\em proper procedures} and {\em function
procedures.} The latter are activated by a function designator as
a constituent of an expression and yield a result that is an operand
of the expression. Proper procedures are activated by a procedure
call. A procedure is a function procedure if its formal parameters
specify a result type. The body of a function procedure must contain
a return statement which defines its result.

All constants, variables, types, and procedures declared within a
procedure body are local to the procedure. Since procedures may be
declared as local objects too, procedure declarations may be nested.
The call of a procedure within its declaration implies recursive activation.

Objects declared in the environment of the procedure are also visible
in those parts of the procedure in which they are not concealed by
a locally declared object with the same name.
{\BNFsize
\begin{verbatim}
  ProcedureDeclaration =
      ProcedureHeading ";" ProcedureBody ident
  ProcedureHeading     =
      PROCEDURE [Receiver] IdentDef [FormalParameters].
  ProcedureBody        =
      DeclarationSequence [ BEGIN StatementSequence ] END.
  DeclarationSequence  =
      { CONST { ConstantDeclaration";" } |
        TYPE { TypeDeclaration ";" } |
        VAR { VariableDeclaration ";" }
      } |
      { ProcedureDeclaration ";" | ForwardDeclaration ";" }.
  ForwardDeclaration   =
      PROCEDURE "^"[Receiver] IdentDef [FormalParameters].
\end{verbatim}}
If a procedure declaration specifies a receiver parameter, the procedure
is considered to be bound to a type (see \ref{o2r:type:bound}).
A forward declaration serves to allow forward references to a procedure
whose actual declaration appears later in the text. The formal parameter
lists of the forward declaration and the actual declaration must {\em match}
(see \ref{o2r:terms}).

\subsection{Formal parameters}\label{o2r:formal:param}

Formal parameters are identifiers declared in the formal parameter
list of a procedure. They correspond to actual parameters specified
in the procedure call. The correspondence between formal and actual
parameters is established when the procedure is called. There are
two kinds of parameters, value and variable parameters, indicated
in the formal parameter list by the absence or presence of the keyword
VAR. Value parameters are local variables to which the value of the
corresponding actual parameter is assigned as an initial value. Variable
parameters correspond to actual parameters that are variables, and
they stand for these variables. The scope of a formal parameter extends
from its declaration to the end of the procedure block in which it
is declared. A function procedure without parameters must have an
empty parameter list. It must be called by a function designator whose
actual parameter list is empty too. The result type of a procedure
can be neither a record nor an array.
{\BNFsize
\begin{verbatim}
  FormalParameters = "(" [FPSection {";"FPSection}] ")"
                     [ ":" Qualident ].
  FPSection        = [VAR] ident {"," ident} ":" Type.
\end{verbatim}}
Let $T_f$ be the type of a formal parameter $f$ (not an open array) and
$T_a$ the type of the corresponding actual parameter $a$. For variable
parameters, $T_a$ must be the {\em same} as $T_f$,
or $T_f$ must be a record type
and $T_a$ an extension of $T_f$. For value parameters, $a$ must be
{\em assignment compatible} with $f$ (see \ref{o2r:terms}).
If $T_f$ is an open array, then $a$ must be {\em array compatible} with $f$
(see \ref{o2r:terms}). The lengths of $f$ are taken from $a$.

\noindent
Examples of procedure declarations:
\begin{verbatim}
  PROCEDURE ReadInt(VAR x: INTEGER);
    VAR i: INTEGER; ch: CHAR;
  BEGIN
    i := 0;
    Read(ch);
    WHILE ("0" <= ch) & (ch >= "9") DO
      i:= 10*i + (ORD(ch)-ORD("0"));
      Read(ch)
    END;
    x := i;
  END ReadInt

  PROCEDURE WriteInt(x: INTEGER);
    (* 0 <= x <= 100000 *)
    VAR i: INTEGER; buf: ARRAY 5 OF INTEGER;
  BEGIN
    i := 0;
    REPEAT
      buf[i] := x MOD 10;
      x := x DIV 10;
      INC(i)
    UNTIL x = 0;
    REPEAT
      DEC(i);
      Write(CHR(buf[i] + ORD("0")))
    UNTIL i = 0;
  END WriteInt

  PROCEDURE WriteString(s: ARRAY OF CHAR);
    VAR i: INTEGER;
  BEGIN
    i := 0;
    WHILE (i < LEN(s)) & (s[i] # 0X) DO
      Write(s[i]);
      INC(i)
    END
  END WriteString

  PROCEDURE log2(x: INTEGER): INTEGER;
    VAR y: INTEGER; (* assume x>0 *)
  BEGIN
    y := 0;
    WHILE x > 1 DO x := x DIV 2; INC(y) END;
    RETURN y
  END log2
\end{verbatim}

\subsection{Type-bound procedures}\label{o2r:type:bound}

Globally declared procedures may be associated with a record type
declared in the same module.
The procedures are said to be {\em bound} to the record type. The
binding is expressed by the type of the {\em receiver} in the heading of
a procedure declaration.  The receiver may be either a variable parameter
of record type $T$ or a value parameter of type {\tt POINTER TO $T$}
(where $T$ is a record type).
The procedure is bound to the type $T$ and is considered local to it.
{\BNFsize
\begin{verbatim}
  ProcedureHeading =
        PROCEDURE [Receiver] IdentDef [FormalParameters].
  Receiver         = "(" [VAR] ident ":" ident ")".
\end{verbatim}}
If a procedure $P$ is bound to a type $T_0$, it is implicitly also bound
to any type $T_1$ which is an extension of $T_0$. However, a procedure
$P'$  (with the same name as $P$) may be explicitly bound to $T_1$ in which
case it overrides the binding of $P$. $P'$ is considered a {\em redefinition}
of $P$ for $T_1$. The formal parameters of $P$ and $P'$ must {\em match}
(see \ref{o2r:terms}).
If $P$ and $T_1$ are exported (see section \ref{o2r:decls})
$P'$ must be exported too.

If $v$ is a designator and $P$ is a type-bound procedure, then $v.P$ denotes
that procedure $P$ which is bound to the dynamic type of $v$ (dynamic
binding). Note, that this may be a different procedure than the one
bound to the static type of $v$. $v$ is passed to $P'$s receiver according
to the parameter passing rules specified in section \ref{o2r:formal:param}.

If $r$ is a receiver parameter declared with type $T$, $r.P$\arrow denotes the
(redefined) procedure $P$ bound to the base type of $T$.

In a forward
declaration of a type-bound procedure the receiver parameter must
be of the {\em same} type as in the actual procedure declaration. The formal
parameter lists of both declarations must {\em match} (\ref{o2r:terms}).

\noindent
Examples:
\begin{verbatim}
  PROCEDURE (t: Tree) Insert (node: Tree);
    VAR p, father: Tree;
  BEGIN
    p := t;
    REPEAT father := p;
      IF node.key = p.key THEN RETURN END;
      IF node.key < p.key THEN p := p.left
      ELSE p := p.right
      END
    UNTIL p = NIL;
    IF node.key < father.key THEN father.left := node
    ELSE father.right := node
    END;
    node.left := NIL;
    node.right := NIL
  END Insert;

  PROCEDURE (t: CenterTree) Insert (node: Tree);
    (*redefinition*)
  BEGIN
    WriteInt(node(CenterTree).width);
    t.Insert^(node)
    (* calls the Insert procedure bound to Tree *)
  END Insert;
\end{verbatim}

\subsection{Predeclared procedures}\label{o2r:predecl:proc}
\index{standard procedures!Oberon-2}

The following table lists the predeclared procedures. Some are generic
procedures, i.e. they apply to several types of operands. v stands
for a variable, x and n for expressions, and T for a type.

\ifonline
\subsection{Function procedures}
\else
\subsubsection{\em Function procedures}
\fi
\label{o2r:std:func}
\index{ABS}\index{ASH}\index{CAP}\index{CHR}
\index{ENTIER}\index{LEN}\index{LONG}
\index{MIN}\index{MAX}\index{ODD}\index{ORD (O2)}\index{SHORT}
\index{SIZE}

\begin{flushleft}
\ifonline
\else
\ifcompact \small \fi
  \newlength{\FuncName}\settowidth{\FuncName}{ENTIER(x)}
  \newlength{\FuncRes}\settowidth{\FuncRes}{LONGREAL}
  \newlength{\FuncArg}
  \newlength{\FuncFun}
  \ifcompact
          \FuncArg=2.5cm
          \FuncFun=3.5cm
  \else
          \FuncArg=2.8cm
          \FuncFun=4.4cm
  \fi
\fi
\ifonline
\begin{tabular}{llll}
\else
\begin{tabular}{p{\FuncName}p{\FuncArg}p{\FuncRes}p{\FuncFun}}
\fi
Name    & Argument type      & Result type & Function \\
        & \\
ABS(x)  & numeric type       & type of x   & absolute value \\
ASH(x,n) & x,n: integer type & LONGINT     & arithmetic shift ($x*2^n$) \\
\ifonline
\else
\end{tabular}
\begin{tabular}{p{\FuncName}p{\FuncArg}p{\FuncRes}p{\FuncFun}}
\fi
CAP(x)  &  CHAR              & CHAR        & x is letter:
                                            corresponding capital letter \\
CHR(x)  &  integer type      & CHAR        & character with ordinal number x \\
ENTIER(x)& real type         & LONGINT     & largest integer not greater than x \\
LEN(v,n) & v: array; n: integer const. & LONGINT
                                           & length of v in dimension n
                                             (first dimension = 0) \\
\ifonline
\else
\end{tabular}
\begin{tabular}{p{\FuncName}p{\FuncArg}p{\FuncRes}p{\FuncFun}}
\fi
LEN(v)   & v: array          & LONGINT     & the same as LEN(v,0) \\
LONG(x)  & SHORTINT          & INTEGER     & identity   \\
         & INTEGER           & LONGINT                  \\
         & REAL              & LONGREAL                 \\
\ifonline
\else
\end{tabular}
\begin{tabular}{p{\FuncName}p{\FuncArg}p{\FuncRes}p{\FuncFun}}
\fi
MAX(T)   & T = basic type    & T           & maximum value of type T \\
         & T = SET           & INTEGER     & maximum element of a set \\
MIN(T)   & T = basic type    & T           & minimum value of type T \\
         & T = SET           & INTEGER     & 0 \\
ODD(x)   & integer type      & BOOLEAN     & x MOD 2 = 1 \\
ORD(x)   & CHAR              & INTEGER     & ordinal number of x \\
\ifonline
\else
\end{tabular}
\begin{tabular}{p{\FuncName}p{\FuncArg}p{\FuncRes}p{\FuncFun}}
\fi
SHORT(x) & LONGINT           & INTEGER     & identity             \\
         & INTEGER           & SHORTINT    & identity             \\
         & LONGREAL          & REAL        & identity (truncation possible) \\
SIZE(T)  & any type          & integer     & number of bytes required by T \\
\end{tabular}
\end{flushleft}

\ifonline
\subsection{Proper procedures}
\else
\subsubsection{\em Proper procedures}
\fi
\label{o2r:std:proc}
\index{COPY}\index{DEC}\index{EXCL}\index{HALT}\index{INC}\index{INCL}
\index{NEW (O2)}

\begin{flushleft}
\ifonline
\else
  \ifcompact \small \fi
  \newlength{\ProcName}\settowidth{\ProcName}{NEW(v,x0,...,xn)}
  \newlength{\ProcArg}
  \newlength{\ProcFun}
  \ifcompact
          \ProcArg=4.0cm
          \ProcFun=3.4cm
  \else
          \ProcArg=4.4cm
          \ProcFun=4.2cm
  \fi
\fi
\ifonline
\begin{tabular}{lll}
\else
\begin{tabular}{p{\ProcName}p{\ProcArg}p{\ProcFun}}
\fi
Name      & Argument types             & Function \\
          & \\
ASSERT(x) & x: Boolean expression      & terminate program execution if not x \\
ASSERT(x,n) & x: Boolean expression; n: integer constant
                                       & terminate program execution if not x \\
\ifonline
\else
\end{tabular}
\begin{tabular}{p{\ProcName}p{\ProcArg}p{\ProcFun}}
\fi
COPY(x,v) & x: character array, string;
           v: character array          &  v := x \\
DEC(v)    & integer type               &  v := v - 1 \\
DEC(v,n)  & v, n: integer type         &  v := v - n \\
\ifonline
\else
\end{tabular}
\begin{tabular}{p{\ProcName}p{\ProcArg}p{\ProcFun}}
\fi
EXCL(v,x) & v: SET; x: integer type    &  v := v - {x} \\
HALT(n)   & integer constant           &  terminate program execution \\
\ifonline
\else
\end{tabular}
\begin{tabular}{p{\ProcName}p{\ProcArg}p{\ProcFun}}
\fi
INC(v)    & integer type               &  v := v + 1 \\
INC(v,n)  & v, n: integer type         &  v := v + n  \\
INCL(v,x) & v: SET; x: integer type    &  v := v + {x} \\
\ifonline
\else
\end{tabular}
\begin{tabular}{p{\ProcName}p{\ProcArg}p{\ProcFun}}
\fi
NEW(v)    & pointer to record or fixed array
                                       &  allocate v\arrow \\
NEW(v,x0,...,xn) &
           v: pointer to open array;
           xi: integer type            & allocate v\arrow with lengths x0...xn \\
\end{tabular}
\end{flushleft}

\noindent
COPY allows the assignment of a string or a character array
containing a terminating 0X to another character array.
If necessary, the assigned value is truncated
to the target length minus one.
The target will always contain 0X as a terminator.
In ASSERT(x,n) and HALT(n),
the interpretation of $n$ is left to the underlying system
implementation.

\section{Modules}\label{o2r:module}

A module is a collection of declarations of constants, types, variables,
and procedures, together with a sequence of statements for the purpose
of assigning initial values to the variables. A module constitutes
a text that is compilable as a unit.
{\BNFsize
\begin{verbatim}
  Module     = MODULE ident ";" [ImportList]
               DeclarationSequence
               [ BEGIN StatementSequence ]
               END ident ".".
  ImportList = IMPORT Import {"," Import} ";".
  Import     = [ident ":="] ident.
\end{verbatim}}
The import list specifies the names of the imported modules. If a
module A is imported by a module M and A exports an identifier x,
then x is referred to as A.x within M. If A is imported as B := A,
the object x is referenced as B.x. This allows short alias names in
qualified identifiers. A module must not import itself.
Identifiers that are to be exported (i.e. that
are to be visible in client modules) must be marked by an export mark
in their declaration (see section \ref{o2r:decls}).

The statement sequence following the symbol BEGIN is
executed when the module is added to a system (loaded), which is done
after the imported modules have been loaded. It follows that cyclic
import of modules is illegal. Individual (parameterless and exported)
procedures can be activated from the system, and these procedures
serve as {\em commands}.
{\small
\begin{verbatim}
MODULE Trees;
(* exports:
  Tree, Node, Insert, Search, Write, NewTree
*)
(* exports read-only: Node.name *)
IMPORT Texts, Oberon;
TYPE
  Tree* = POINTER TO Node;
  Node* = RECORD
    name-: POINTER TO ARRAY OF CHAR;
    left, right: Tree
  END;

VAR w: Texts.Writer;

PROCEDURE (t: Tree) Insert* (name: ARRAY OF CHAR);
  VAR p, father: Tree;
BEGIN
  p := t;
  REPEAT father := p;
    IF name = p.name^ THEN RETURN END;
    IF name < p.name^ THEN p := p.left
    ELSE p := p.right
    END
  UNTIL p = NIL;
  NEW(p); p.left := NIL; p.right := NIL;
  NEW(p.name,LEN(name)+1);
  COPY(name,p.name^);
  IF name < father.name^ THEN father.left := p
  ELSE father.right := p
  END;
END Insert;

PROCEDURE (t: Tree) Search* (name: ARRAY OF CHAR): Tree;
  VAR p: Tree;
BEGIN
  p := t;
  WHILE (p # NIL) & (name # p.name^) DO
    IF name = p.name^ THEN p := p.left
    ELSE p := p.right
    END
  END;
  RETURN p
END Search;

PROCEDURE (t: Tree) Write*;
BEGIN
  IF t.left # NIL THEN t.left.Write END;
  Texts.WriteString(w, t.name^);
  Texts.WriteLn(w);
  Texts.Append(Oberon.Log, w.buf);
  IF t.right # NIL THEN t.right.Write END
END Write;

PROCEDURE NewTree* (): Tree;
  VAR t: Tree;
BEGIN
  NEW(t); NEW(t.name, 1);
  t.name[0] := 0X;
  t.left := NIL; t.right := NIL;
  RETURN t
END NewTree;

BEGIN
  Texts.OpenWriter(w)
END Trees.
\end{verbatim}
}

\section{Definition of terms}\label{o2r:terms}

\begin{tabular}{ll}
{\bf Integer types}  &       SHORTINT, INTEGER, LONGINT \\
{\bf Real types}     &       REAL, LONGREAL \\
{\bf Numeric types}  &       integer types, real types \\
\end{tabular}

\ifonline
\subsection{Same types}
\else
\subsubsection{Same types}
\fi

Two variables $a$ and $b$ with types $T_a$ and $T_b$ are of
the {\em same} type if
\begin{enumerate}
\item
$T_a$ and $T_b$ are both denoted by the same type identifier,
or
\item
$T_a$ is declared to equal $T_b$ in a type declaration
of the form $T_a = T_b$, or
\item
$a$ and $b$ appear in the same identifier list in a
variable, record field, or formal parameter declaration and are not
open arrays.
\end{enumerate}

\ifonline
\subsection{Equal types}
\else
\subsubsection{Equal types}
\fi

Two types $T_a$ and $T_b$ are {\em equal} if
\begin{enumerate}
\item
$T_a$ and $T_b$ are the {\em same} type, or
\item
$T_a$ and $T_b$ are open array types with {\em equal} element types, or
\item
$T_a$ and $T_b$ are procedure types whose formal parameter lists {\em match}.
\end{enumerate}

\ifonline
\subsection{Type inclusion}
\else
\subsubsection{Type inclusion}
\fi

Numeric types {\em include} (the values of) smaller numeric types according
to the following hierarchy
$$
\mbox{LONGREAL} \supseteq
\mbox{REAL} \supseteq
\mbox{LONGINT} \supseteq
\mbox{INTEGER} \supseteq
\mbox{SHORTINT}
$$

\ifonline
\subsection{Type extension (base type)}
\else
\subsubsection{Type extension (base type)}
\fi

Given a type declaration {\tt $T_b$ = RECORD ($T_a$)... END},
$T_b$ is a {\em direct extension} of $T_a$,
and $T_a$ is a {\em direct base type} of $T_b$. A type $T_b$ is
an {\em extension} of a type $T_a$ ($T_a$ is a {\em base type} of $T_b$) if
\begin{enumerate}
\item
$T_a$ and $T_b$ are the {\em same} types, or
\item
$T_b$ is a {\em direct extension} of an {\em extension} of $T_a$
\end{enumerate}

If {\tt $P_a$ = POINTER TO $T_a$}
and {\tt $P_b$ = POINTER TO $T_b$}, $P_b$ is an {\em extension} of
$P_a$ ($P_a$ is a {\em base type} of $P_b$)
if $T_b$ is an {\em extension} of $T_a$.

\ifonline
\subsection{Assignment compatible}
\else
\subsubsection{Assignment compatible}
\fi

An expression $e$ of type $T_e$ is {\em assignment compatible} with a variable
$v$ of type $T_v$ if one of the following conditions hold:
\begin{enumerate}
\item
$T_e$ and $T_v$ are the {\em same} type;
\item
$T_e$ and $T_v$ are numeric types and $T_v$ {\em includes} $T_e$;
\item
$T_e$ and $T_v$ are record types and $T_e$ is an {\em extension}
of $T_v$ and the dynamic type of $v$ is $T_v$ ;
\item
$T_e$ and $T_v$ are pointer types and $T_e$ is an {\em extension}
of $T_v$;
\item
$T_v$ is a pointer or a procedure type and $e$ is NIL;
\item
$T_v$ is ARRAY n OF CHAR, $e$ is a string constant with
$m$ characters, and $m < n$;
\item
$T_v$ is a procedure type and $e$ is the name of a procedure
whose formal parameters {\em match} those of $T_v$.
\end{enumerate}

\ifonline
\subsection{Array compatible}
\else
\subsubsection{Array compatible}
\fi

An actual parameter $a$ of type $T_a$ is {\em array compatible} with a formal
parameter $f$ of type $T_f$ if
\begin{enumerate}
\item
$T_f$ and $T_a$ are the {\em same} type, or
\item
$T_f$ is an open array, $T_a$ is any array, and their
element types are {\em array compatible}, or
\item
$T_f$ is ARRAY OF CHAR and $a$ is a string.
\end{enumerate}

\ifonline
\subsection{Expression compatible}
\else
\subsubsection{Expression compatible}
\fi

For a given operator, the types of its operands are {\em expression compatible}
if they conform to the following table (which shows also the result
type of the expression). Character arrays to be compared
must contain 0X as a terminator. Type $T_1$ must be an extension
of type $T_0$:
\begin{flushleft}
\ifonline
\else
  \ifcompact \footnotesize \else \small \fi
  \newlength{\Operator}\settowidth{\Operator}{\verb|= = < <= > >=|}
  \newlength{\Operand}
  \newlength{\ResType}
  \ifcompact
          \Operand=2.4cm
          \ResType=2.7cm
  \else
          \Operand=2.7cm
          \ResType=3.7cm
  \fi
\fi
\ifonline
\begin{tabular}{llll}
\else
\begin{tabular}{p{\Operator}p{\Operand}p{\Operand}p{\ResType}}
\fi
operator & first operand   & second operand     & result type \\
\verb|+ - *| & numeric         &  numeric
                               & smallest numeric type including both operands \\
\verb|/|     & numeric         & numeric
                               & smallest real type including both operands \\
\verb|+ - * /|  & SET           & SET                &  SET \\
\ifonline
\else
\end{tabular}
\begin{tabular}{p{\Operator}p{\Operand}p{\Operand}p{\ResType}}
\fi
DIV MOD  & integer         & integer
                                & smallest integer type including both operands \\
OR \verb|&| \verb|~| & BOOLEAN & BOOLEAN        & BOOLEAN \\
\ifonline
\else
\end{tabular}
\begin{tabular}{p{\Operator}p{\Operand}p{\Operand}p{\ResType}}
\fi
\verb|= # < <= > >=|
              & numeric         &   numeric     & BOOLEAN \\
              & CHAR            &  CHAR         & BOOLEAN \\
              & character array, string
              & character array, string         & BOOLEAN \\
\verb|= #|    & BOOLEAN         & BOOLEAN       & BOOLEAN \\
              & SET             &     SET       & BOOLEAN \\
              & NIL, POINTER TO T0 or T1
              & NIL, POINTER TO T0 or T1        & BOOLEAN \\
              & procedure type T, NIL
              & procedure type T, NIL           & BOOLEAN \\
IN            & integer         &     SET       & BOOLEAN \\
IS            & pointer         &     pointer   & BOOLEAN \\
              & record          &     record    & BOOLEAN \\
\end{tabular}
\end{flushleft}

\subsection{Matching formal parameter lists}

Two formal parameter lists {\em match} if
\begin{enumerate}
\item
they have the same number of parameters, and
\item
they have either the {\em same} function result type or none, and
\item
parameters at corresponding positions have {\em equal} types, and
\item
parameters at corresponding positions are both either
value or variable parameters.
\end{enumerate}

\section{Syntax of Oberon-2}

{
\ifcompact \small \fi
\begin{verbatim}
Module      = MODULE ident ";" [ImportList] DeclSeq
             [BEGIN StatementSeq] END ident ".".
ImportList  = IMPORT [ident ":="] ident
              {"," [ident ":="] ident} ";".
DeclSeq     =  { CONST {ConstDecl ";" }
            | TYPE {TypeDecl ";"}
            | VAR {VarDecl ";"}} {ProcDecl ";"
            | ForwardDecl ";"}.
ConstDecl   = IdentDef "=" ConstExpr.
TypeDecl    = IdentDef "=" Type.
VarDecl     = IdentList ":" Type.
ProcDecl    = PROCEDURE [Receiver] IdentDef
              [FormalPars] ";" DeclSeq [BEGIN
              StatementSeq] END ident.
ForwardDecl = PROCEDURE "^" [Receiver] IdentDef
              [FormalPars].
FormalPars  = "(" [FPSection {";" FPSection}] ")"
              [":" Qualident].
FPSection   = [VAR] ident {"," ident} ":" Type.
Receiver    = "(" [VAR] ident ":" ident ")".
Type        = Qualident
            | ARRAY [ConstExpr {"," ConstExpr}] OF Type
            | RECORD ["("Qualident")"] FieldList
              {";" FieldList} END
            | POINTER TO Type
            | PROCEDURE [FormalPars]
FieldList   = [IdentList ":" Type].
StatementSeq= Statement {";" Statement}.
Statement   = [Designator ":=" Expr
            | Designator ["(" [ExprList] ")"]
            | IF Expr THEN StatementSeq
             {ELSIF Expr THEN StatementSeq}
             [ELSE StatementSeq] END
            | CASE Expr OF Case {"|" Case}
              [ELSE StatementSeq] END
            | WHILE Expr DO StatementSeq END
            | REPEAT StatementSeq UNTIL Expr
            | FOR ident ":=" Expr TO Expr
              [BY ConstExpr] DO StatementSeq END
            | LOOP StatementSeq END
            | WITH Guard DO StatementSeq
             {"|" Guard DO StatementSeq}
              [ELSE StatementSeq] END
            | EXIT
            | RETURN [Expr]].
Case        = [CaseLabels {"," CaseLabels}
              ":" StatementSeq].
CaseLabels  = ConstExpr [".." ConstExpr].
Guard       = Qualident ":" Qualident.
ConstExpr   = Expr.
Expr        = SimpleExpr [Relation SimpleExpr].
SimpleExpr  = ["+" | "-"] Term {AddOp Term}.
Term        = Factor {MulOp Factor}.
Factor      = Designator ["(" [ExprList] ")"]
            | number | character | string | NIL
            | Set | "(" Expr ")" | " ~ " Factor.
Set         = "{" [Element {"," Element}] "}".
Element     = Expr [".." Expr].
Relation    = "=" | "#" | "<" | "<=" | ">" | ">="
            | IN | IS.
AddOp       = "+" | "-" | OR.
MulOp       = " * " | "/" | DIV | MOD | "&".
Designator  = Qualident {"." ident | "[" ExprList "]"
            |  "^" | "(" Qualident ")"}.
ExprList    = Expr {"," Expr}.
IdentList   = IdentDef {"," IdentDef}.
Qualident   = [ident "."] ident.
IdentDef    = ident [" * " | "-"].
\end{verbatim}
}

\section{The module SYSTEM}\label{o2r:SYSTEM}

The module SYSTEM contains certain types and procedures that are necessary
to implement {\em low-level} operations particular to a given computer and/or
implementation. These include for example facilities for accessing
devices that are controlled by the computer, and facilities to break
the type compatibility rules otherwise imposed by the language definition.
It is strongly recommended to restrict their use to specific modules
(called {\em low-level} modules). Such modules are inherently non-portable,
but easily recognized due to the identifier SYSTEM appearing in their
import list. The following specifications hold for the implementation
of Oberon-2 on the Ceres computer.

Module SYSTEM exports a type BYTE with the following characteristics:
Variables of type CHAR or SHORTINT can be assigned to variables of
type BYTE. If a formal variable parameter is of type ARRAY OF BYTE
then the corresponding actual parameter may be of any type.

Another type exported by module SYSTEM is the type PTR. Variables
of any pointer type may be assigned to variables of type PTR. If a
formal variable parameter is of type PTR, the actual parameter may
be of any pointer type.

The procedures contained in module SYSTEM are listed in the following
tables. Most of them correspond to single instructions compiled as
in-line code. For details, the reader is referred to the processor
manual. $v$ stands for a variable, $x$, $y$, $a$, and $n$ for expressions, and
$T$ for a type.

\ifonline
\subsection{Function procedures}
\else
\subsubsection{Function procedures}
\fi

\begin{flushleft}
\ifonline
\else
  \ifcompact \small \fi
  \settowidth{\FuncName}{ROT(w,w)}
  \settowidth{\FuncRes}{BOOLEAN}
  \ifcompact
          \FuncArg=2.5cm
          \FuncFun=3.5cm
  \else
          \FuncArg=2.8cm
          \FuncFun=4.4cm
  \fi
\fi
\ifonline
\begin{tabular}{llll}
\else
\begin{tabular}{p{\FuncName}p{\FuncArg}p{\FuncRes}p{\FuncFun}}
\fi
Name     & Argument types   & Result type & Function \\
         & \\
ADR(v)   & any              & LONGINT     & address of variable v \\
BIT(a,n) & a: LONGINT
          n: integer        & BOOLEAN     & bit n of Mem[a] \\
CC(n)    & integer constant & BOOLEAN     & condition $n$ ($0\leq n\leq 15$) \\
\ifonline
\else
\end{tabular}
\begin{tabular}{p{\FuncName}p{\FuncArg}p{\FuncRes}p{\FuncFun}}
\fi
LSH(x,n) & x: integer,
          CHAR, BYTE;
          n: integer         & type of x  &  logical shift \\
ROT(x,n) & x: integer,
          CHAR, BYTE;
          n: integer         & type of x  &  rotation \\
VAL(T,x) & T, x: any type    & T          & x interpreted as of type T \\
\end{tabular}
\end{flushleft}
\index{SYSTEM!ADR (O2)}
\index{SYSTEM!BIT}\index{SYSTEM!CC}\index{SYSTEM!LSH}
\index{SYSTEM!ROT}\index{SYSTEM!VAL}

\ifonline
\subsection{Proper procedures}
\else
\subsubsection{Proper procedures}
\fi

\begin{flushleft}
\ifonline
\else
  \ifcompact \small \fi
  \settowidth{\ProcName}{MOVE(a0,a1,n)}
  \ifcompact
          \ProcArg=4.0cm
          \ProcFun=3.4cm
  \else
          \ProcArg=4.4cm
          \ProcFun=4.2cm
  \fi
\fi
\ifonline
\begin{tabular}{lll}
\else
\begin{tabular}{p{\ProcName}p{\ProcArg}p{\ProcFun}}
\fi
Name      & Argument types             & Function \\
          & \\
GET(a,v)    & a: LONGINT;
              v: any basic type,
              pointer, procedure type & v := Mem[a] \\
PUT(a,x)    &  a: LONGINT;
              x: any basic type,
              pointer, procedure type & Mem[a] :=x \\
\ifonline
\else
\end{tabular}
\begin{tabular}{p{\ProcName}p{\ProcArg}p{\ProcFun}}
\fi
GETREG(n,v) &  n: integer constant;
              v: any basic type,
              pointer, procedure type & v := Register n \\
PUTREG(n,x) &  n: integer constant;
              x: any basic type,
              pointer, procedure type & Register n := x \\
\ifonline
\else
\end{tabular}
\begin{tabular}{p{\ProcName}p{\ProcArg}p{\ProcFun}}
\fi
MOVE(a0,a1,n) &  a0, a1: LONGINT;
                 n: integer           & Mem[a1..a1+n-1] := Mem[a0..a0+n-1] \\
NEW(v,n)      & v: any pointer;
              n: integer              & allocate storage block of n bytes
                                       assign its address to v \\
\end{tabular}
\end{flushleft}
\index{SYSTEM!GET}
\index{SYSTEM!PUT}
\index{SYSTEM!GETREG}
\index{SYSTEM!PUTREG}
\index{SYSTEM!MOVE}
\index{SYSTEM!NEW (O2)}

\printindex

\end{document}
